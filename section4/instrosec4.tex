Diboson production processes provide a very good sensitivity to a large set of new-physics effects and can be effectively used
to test interesting classes of BSM theories. 

The study of longitudinally-polarized dibosons production in the high-energy limit $E\gg m_{W}$ is greatly simplified
by using the Equivalence Theorem~\cite{Chanowitz:1985hj,Wulzer:2013mza}.
In this formalism, external longitudinally-polarized vector states are represented in Feynman diagrams as the corresponding
scalar Goldstone bosons, up to corrections of order $m_W/E$ from diagrams with gauge external lines.
%Furthermore, the $E\gg m_W$ limit can be safely taken in the internal line propagators and in the vertices, making that
%all the effects (masses and vertices) induced by the Higgs VEV manifestly produce order $m_W/E$ corrections.
In order to assess the leading energy behavior, it is sufficient to study the amplitude in the unbroken phase,
where the EW bosons are massless and the $G_{\rm{SM}}={\textrm{SU}}(2)_L\times{\textrm{U}}(1)_Y$ symmetry is exact.
Given that the Goldstone bosons live in the Higgs doublet $H$, together with the Higgs particle, $G_{\rm{SM}}$ implies
that the high-energy behavior of the former ones are connected with the latter. This is the reason why $V_LV_L$
and $V_Lh$ production processes, collectively denoted as $\Phi\Phi'$ in what follows, should be considered together.

{\color{red} MORE}
