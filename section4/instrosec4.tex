An important aspect of both the HL and HE programs, is their enhanced sensitivity to the high-energy tails of kinematic distributions.  
These constitute genuinely \emph{new} observables with which we can realistically  conceive \emph{high-energy precision tests} that have been impossible at previous experimental facilities.
 There are two reasons that make this precision program appealing. The first is that we can \emph{define} the high-energy region as that in which statistical uncertainty becomes comparable to systematic uncertainty: in this way, high-energy precision probes are, by construction, expected to deliver the fastest relative improvement with accumulated luminosity, contrary to other observables which are, sooner or later, saturated by systematics.

The second reason why these tests are particular interesting is that the very framework in which precision tests are conceived, that of Effective Field Theories (EFTs) - {\color{red}  see section XXX ...Will we have an EFT section in 2?}-  implies effects that grow with the energy $E$. In particular, at the dimension-6 level, the expected growth is $\propto E^2$, implying a quadratic enhancement of the signal. 
As an order of magnitude estimate, an LHC $O(10\%)$ measurement at 1 TeV, is equivalent in precision to a $O(0.1\%)$ measurement at LEP (where at the $Z$-pole $E\approx 100$ GeV). For this reason, the HL-LHC (and even more so its HE upgrade) constitute an important continuation of the precision program. 

In this chapter, we provide a perspective on the importance of these high-energy probes, and collect a number of contributions that target energy-growing effects in the EFT framework.
Our focus is of course Higgs-physics; yet, in the  high-energy  limit $E\gg m_{W}$, the longitudinal polarisations of gauge bosons are also associated with the scalar sector, as can be understood by the Equivalence Theorem~\cite{Chanowitz:1985hj,Wulzer:2013mza}, where external longitudinally-polarized vector states are represented in Feynman diagrams as the corresponding
scalar Goldstone bosons, up to corrections of order $m_W/E$ from diagrams with gauge external lines.
This brings us to study, in wider generality, processes involving gauge  and Higgs bosons. %We pay particular attention to dibosons, which display the largest cross-sections.

Sections~\ref{WZlong} and \ref{sec:ZHWZeft} discuss the reach to modified Higgs sectors from $WZ$, $WH$ and $ZH$ high-$p_T$
distributions, while section~\ref{sec:WZtrans} focusses on an additional class of effects, associated with new physics in the transverse polarisations of vectors, that also modifies $WZ$ processes. 
Drell-Yann processes also constitute a very clean and powerful probe for new physics coupling to the electroweak bosons, as we discuss in section~\ref{DY}. In sections \ref{sect-illus} and \ref{sec:VVHHcont} we motivate and study modifications of the $hhWW$ coupling, that can be tested in VBF processes. Finally, a more complete EFT discussion of the VBF topology appears in section~\ref{sec:VBFdim6eft} and in section \ref{sect-ssWW} for the same-sign case.












