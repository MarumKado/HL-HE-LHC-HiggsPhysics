We initiate a phenomenological study of ``universal relations'' in composite Higgs models, which  are  dictated by nonlinear shift symmetries acting on the 125 GeV Higgs boson. These are relations among one Higgs couplings with two electroweak gauge bosons (HVV), two Higgses couplings with two electroweak gauge bosons (HHVV), one Higgs couplings with three electroweak gauge bosons (HVVV), as well as triple gauge boson couplings (TGC), which are all controlled by a single input parameter: the decay constant $f$ of the pseudo-Nambu-Goldstone Higgs boson.  Assuming custodial invariance in strong sector, the  relation is independent of the symmetry breaking pattern in the UV, for an arbitrary  symmetric coset $G/H$. The complete list of corrections to HVV, HHVV, HVVV and TGC couplings in composite Higgs models is presented to all orders in $1/f$, and up to four-derivative level,  without referring to a particular $G/H$. We then present several examples of  universal relations in ratios of coefficients which could be extracted experimentally. Measuring the universal relation requires a precision sensitive to effects of dimension-8 operators in the effective Lagrangian and highlights the importance of verifying the tensor structure of HHVV interactions in the standard model, which remains untested to date.
%%%%%%%%%%%%%%%%%%%%%%%%%
Acknowledgments: 
%%%%%%%%%%%%%%%%%%%%%%%%%
This work is supported in part by the U.S. Department of Energy under contracts No. DE-AC02-06CH11357 and No. DE-SC0010143.
%%%%%%%%%%%%%%%%%%%%%%%%%

