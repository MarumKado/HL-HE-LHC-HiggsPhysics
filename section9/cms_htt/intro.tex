%TO DO: 
%        -remove 35.9 from plots, replace by reference
%        -add model-indep plot for 3000 comparing S1/S2/stat-only
%        indep plots: /afs/hephy.at/work/m/mflechl/cmssw/mssm_projection/CMSSW_7_4_7/src/CombineHarvester/MSSMFull2016/output
%        mssm plots : /afs/cern.ch/work/m/mflechl/mssm_asymgrid/CMSSW_7_4_7/src/CombineHarvester/MSSMFull2016
%
%\section{Introduction}
%\label{sec:intro}
%The discovery of a Higgs boson was announced by the ATLAS and CMS collaborations in 2012~\cite{ATLASHIGGS2012,CMSHIGGS2012,CMSHIGGSJHEP}. 
%Since then a large number of precision measurements have shown the new particle to be compatible with standard model (SM) predictions. 
%However, there are many arguments in favor of theories which predict the existence of additional heavy Higgs bosons. One such theory is 
%supersymmetry~\cite{Golfand:1971iw,Wess:1974tw}. The minimal supersymmetric extension of the SM (MSSM)~\cite{Fayet:1974pd,Fayet:1977yc} 
%predicts two Higgs doublets which leads to the existence of five Higgs bosons: a light (\Ph) and a heavy (\PH) scalar, 
%a pseudoscalar (\PA), and a charged pair ($\PH^{\pm}$). 
Analyses which search for MSSM Higgs bosons have been performed using the 2016 dataset with 35.9\fbinv of data and have not shown any excess above the 
SM background~\cite{HIG-18-014,HIG-16-018,HIG-17-020}.
However, so far the LHC has only collected a small fraction of its lifetime target integrated luminosity and
several future measurements are sensitive to models beyond the SM not excluded by the current data. In particular 
analyses where the sensitivity is limited by statistical uncertainties will see a significantly extended reach in the 
coming years. Among these are searches for MSSM Higgs bosons.

The projection of the search for heavy Higgs bosons decaying into a pair of tau leptons to higher
luminosities is documented in detail in the following. The projection is based on the most recent CMS 
publication of this analysis, performed using 35.9\fbinv of data collected during 2016, at a centre-of-mass
energy of 13$\TeV$~\cite{HIG-17-020}, referred to as HIG-17-020 in the following. 
In HIG-17-020, all the details of the analysis, including simulation samples, background estimation methods,
systematic uncertainties, and different interpretations are described. 
Only details of direct relevance to the projection are documented in the following.

The analysis is a direct search for a resonance decaying to two tau leptons.
The following tau lepton decay mode combinations, called channels, are considered: $\mutau$, $\etau$,
$\tautau$, and $\emu$, where $\tauhad$ indicates a tau lepton decaying hadronically. In
all of these channels, events are separated into those which contain
at least one b-tagged jet and those which do not contain any b-tagged jets. 
The goal of this categorization is to increase the sensitivity to the dominant MSSM production modes, 
gluon fusion (ggH) and production in association with b quarks (bbH). The final discriminant is the total 
transverse mass, as defined in Ref.~\cite{HIG-17-020}.
%\begin{equation}
%m_{\text{T}}^{\text{tot}}
%= \sqrt{ {m_{\text{T}}^{2}(\ETmiss,\tau_{1}^{\text{vis}})} +
%{m_{\text{T}}^{2}(\ETmiss,\tau_{2}^{\text{vis}})} +
%{m_{\text{T}}^{2}(\tau_{1}^{\text{vis}},\tau_{2}^{\text{vis}} )}  }\,,
%\end{equation}
The signal hypotheses considered consist of additional Higgs
bosons in the mass range of 90$\GeV$ to 3.2$\TeV$.
% Results are interpreted both in terms of model-independent limits on the two signal production processes and
%model-dependent limits in MSSM benchmark scenarios.

A projection of this analysis is carried out by scaling all the signal
and background processes to an integrated luminosity of 3000\fbinv, expected to be collected at the high-luminosity LHC (HL-LHC). 
The upgrade and the expected performance of the CMS detector are described in detail
in the Technical Proposal and the Technical Design Reports for the Phase-II Upgrade of the CMS 
Detector~\cite{CMSPhase2TP,CMSPhase2TrackerTDR,CMSPhase2BarrelTDR,CMSPhase2MuonTDR,CMSPhase2EndcapTDR}. 
The most recent CMS projection of the sensitivity to MSSM Higgs boson decays to a pair of tau leptons is reported 
in Ref.~\cite{FTR-16-002}.
%Special care is taken to use realistic assumptions for the development of systematic uncertainties at high luminosity. 
The results are presented in terms of model-independent limits on a heavy resonance 
decaying to two tau leptons, as well as interpreted in MSSM benchmark scenarios.
