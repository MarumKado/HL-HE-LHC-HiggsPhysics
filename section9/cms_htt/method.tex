\paragraph{Projection methodology}
\label{sec:method}
%
%The projection is based on the published results of the 2016 analysis which uses a dataset corresponding to an integrated luminosity of 35.9\fbinv. 
%These results are scaled to the target luminosity (300\fbinv or 3000\fbinv).
Three scenarios are considered to project systematic uncertainties:
\begin{itemize}
\item
Statistical uncertainties-only: All systematic uncertainties are neglected.
\item
Run 2 systematic uncertainties: All systematic uncertainties are constant with respect to luminosity, i.e. they are the same as for the 2016 results.
\item
YR18 systematic uncertainties: Systematic uncertainties improve with integrated luminosity, following a set of assumptions explained below.
\end{itemize}
%The statistical interpretation is extended in two ways. 
%Firstly, by introducing a multiplicative factor 
%which equally scales all signal and background processes, acting as a scale factor between the luminosity used in the 2016 analysis (35.9\fbinv) and 
%the target luminosity (300\fbinv or 3000\fbinv).
%Secondly, by scaling selected systematic uncertainties with luminosity, in some case with an individual minimum value. 
For the YR18 scenario, selected systematic uncertainties are scaled down as a function of luminosity until they reach a certain minimum value. 
Specifically, all pre-fit uncertainties of experimental nature (including the statistical uncertainty in control regions and of simulation samples) 
are scaled proportionally to the square root of the integrated luminosity.
The following minimum values are enforced:
\begin{itemize}
\item
muon efficiency: 25\% of the 2016 value, corresponding to an average absolute uncertainty of about 0.5\%; 
\item
electron, $\tauhad$, and b-tagging efficiency: 50\% of the 2016 value, 
corresponding to an average absolute uncertainty of about 0.5\%, 2.5\%, and 1.0\%, respectively.
\item
estimate of the background due to jets misreconstructed as $\tauhad$ using a fake factor method~\cite{HIG-15-007}, 
for the subset of its uncertainties which is not of a statistical nature: 50\% of the 2016 values.
\item
luminosity uncertainty: minimum value 1\%.
\item
theory uncertainties: halved with respect to the 2016 values, independently of the luminosity for all projections.
\end{itemize}
Note that for limits where the Higgs boson mass is above about 1 TeV, the statistical uncertainties are dominant and the choice of systematic uncertainty values has 
a negligible impact on the result.

The lightest Higgs boson \Ph is excluded from the SM versus MSSM hypothesis test for the following reason: 
With increasing luminosity, the search will become sensitive to this boson. However, the current benchmark scenarios do not 
incorporate the properties of the \Ph boson with the accuracy required at the time of the HL-LHC. 
Certainly the benchmark scenarios will evolve with time in this respect. Therefore the signal hypothesis includes 
only the heavy $\PA$ and \PH bosons, to demonstrate the search potential only for these.
%
%At high integrated luminosities the MSSM limits become sensitive to the small differences 
%predicted in properties of the already discovered Higgs boson with a mass of 125 GeV. Thus 
%if the lightest Higgs boson, h, is included in the SM versus MSSM hypothesis test, 
%this would result in a complete expected exclusion of the $m_{\PA}$-$\tan\beta$ parameter space, including high $m_{\PA}$ values, 
%due to slightly altered rates of h production with respect to the SM prediction.
%This exclusion would be artificial since as properties of the 125$\,\UGeV$ Higgs boson are measured 
%to increasingly high precision in the future,
%the MSSM benchmarks are expected to be updated to retain
%consistency with the experimental observation. To avoid this, 
%only the presence of the two additional neutral 
%Higgs bosons H and A is tested.
%%, as comparing the SM and MSSM predictions for the
%%125 GeV Higgs boson would yield a complete, yet unrealistic, exclusion of the
%%$m_{\PA}$-$\tan\beta$ parameter space. As integrated luminosity increases and the properties
%%of the 125$\,\UGeV$ Higgs boson are measured to increasingly high precision,
%%the MSSM benchmarks are expected to be updated to retain
%%consistency with the experimental observation.
%This procedure ensures that all exclusion power for a high-mass Higgs boson comes 
%entirely from the high-mass region.
