Searches for Minimal Supersymmetric Standard Model  Higgs bosons have been performed by CMS using
the 2016 data from the LHC Run~2~\cite{HIG-18-014,HIG-16-018,HIG-17-020}.
So far,
no significant evidence for physics beyond the SM has been found.
However, the LHC to date has delivered only a small fraction of the
integrated luminosity expected over its lifetime.
%In 2012, a Higgs boson was discovered by the ATLAS and CMS collaborations~\cite{ATLASHIGGS2012,CMSHIGGS2012,CMSHIGGSJHEP}. 
%Since then, a large number of precise measurements of the properties of this particle as well as searches for additional Higgs 
%bosons have been performed with LHC Run-I and Run-II data. So far, all are in agreement with the standard model (SM) expectations within uncertainties. 
%However, up to now the LHC has only collected a small fraction of its lifetime target integrated luminosity and
Searches that are currently limited by statistical precision
will see significant extensions in their reach as larger data sets are collected.
Among the searches that will benefit are those for MSSM Higgs bosons.

In this section,
projections are presented for the reach that can be expected at higher luminosities
in searches for heavy neutral Higgs bosons that decay to a pair of tau leptons~\cite{CMS-PAS-FTR-18-017}.
The projections are based on the most recent CMS publication
for this search~\cite{HIG-17-020},
performed using 35.9\fbinv of data collected during 2016
at a center-of-mass energy of 13 TeV.
%In the following,
%this search is referred to as HIG-17-020. 
All the details of the analysis,
including the simulated event samples, background estimation methods,
systematic uncertainties, and different interpretations are described in Ref.~\cite{HIG-17-020}.
Only details of direct relevance to the projection are presented here.

The analysis is a direct search for a neutral resonance
decaying to two tau leptons.
The following tau lepton decay mode combinations are considered: $\mutau$, $\etau$,
$\tautau$, and $\emu$, where $\tauhad$ denotes a hadronically decaying tau lepton.
In all these channels, events are separated into those that contain
at least one b-tagged jet and those that do not contain any b-tagged jet. 
The goal of this categorization is to increase sensitivity to the dominant MSSM production modes: 
gluon fusion (ggH) and production in association with b quarks (bbH).
The final discriminant is the total 
transverse mass, defined in Ref.~\cite{HIG-17-020}.
%\begin{equation}
%m_{\text{T}}^{\text{tot}}
%= \sqrt{ {m_{\text{T}}^{2}(\ETmiss,\tau_{1}^{\text{vis}})} +
%{m_{\text{T}}^{2}(\ETmiss,\tau_{2}^{\text{vis}})} +
%{m_{\text{T}}^{2}(\tau_{1}^{\text{vis}},\tau_{2}^{\text{vis}} )}  }\,,
%\end{equation}
The signal hypotheses considered consist of additional Higgs
bosons in the mass range from 90 GeV to 3.2 TeV.
The projection of the limits %obtained in the HIG-17-020 analysis
is performed by scaling all the signal
and background processes to integrated luminosities
of 300 and 3000\fbinv,
where the latter integrated luminosity corresponds to the total that is
expected for the High-Luminosity LHC. 

A previous CMS projection of the sensitivity for
MSSM Higgs boson decays to a pair of tau leptons at the HL-LHC
is reported in Ref.~\cite{FTR-16-002}.
The results are presented in terms of model independent
limits on a heavy resonance (either \PH\ or {\PA},
generically referred to as \PH\ below)
decaying to two tau leptons,
and are also interpreted in the context of MSSM benchmark scenarios.