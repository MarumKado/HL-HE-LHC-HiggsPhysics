\medskip
\begin{center}
{Carlos Bautista,$^{1,2}$ Leonardo de Lima,$^3$ Ricardo D'Elia Matheus,$^1$ Eduardo Pont\'on,$^{1,2}$ Le\^onidas A. Fernandes do Prado,$^{1,4}$ and Aurore Savoy-Navarro$^4$}

{\small 
$^1$Instituto de F\'isica Te\'orica -- UNESP, S\~ao Paulo, Brazil \\
$^2$ICTP South American Institute for Fundamental Research, S\~ao Paulo, Brazil \\
$^3$Universidade Federal da Fronteira Sul, Realeza, Brazil \\
$^4$IRFU-CEA, Universit\'e Paris-Saclay and CNRS-IN2P3, France}
\end{center}
\label{sec9:MCHMtthh}
%\medskip
%\noindent
%\textit{\bf Abstract:}
%We consider the production of one or two Higgs bosons in
%association with a top/anti-top pair in the context of Composite Higgs
%scenarios.  Our focus is on the minimal composite Higgs model, based
%on the symmetry breaking pattern $SO(5) \to SO(4)$.  In the top sector
%we consider two possibilities: fermion resonances in the fundamental
%$\bf 5$ representation of $SO(5)$ and in the symmetric ${\bf 14}$
%representation.  We simulate the ${t\bar t}h$ and ${\bar t}thh$
%processes in MadGraph, and present first results relevant for the
%14~TeV High Luminosity phase of the LHC, and the possible High Energy
%LHC upgrade at 27~TeV. Our main focus is on the non-resonant
%production of the previous final states.
%%%%%%%%%%%%%%%%%%%%%%%%%%%%%%%%%%%%%
\subsubsection{Introduction}
%%%%%%%%%%%%%%%%%%%%%%%%%%%%%%%%%%%%%

With the discovery of the Higgs boson~\cite{Aad:2012tfa,
CMSHiggsJuly2012} the question of whether this resonance is a
composite state, or rather the first observed scalar particle that
appears elementary down to distances much shorter than its Compton
wavelength, has gained new prominence.  We consider the question of
Higgs compositeness~\cite{Kaplan:1983fs, Kaplan:1983sm, Georgi:1984ef, Georgi:1984af, Dugan:1984hq} and the possible effects on the $t\bar{t}h$ and
$t\bar{t}  h h$ processes.  The first one has already been
observed~\cite{Aaboud:2018urx, Sirunyan:2018hoz}, and is consistent
with the SM expectation, although with large uncertainties of order
20\%.  The second one is of particular interest for the models we
consider, since one expects new charge 2/3 vectorlike ``top partners"
that can decay in the $th$ channel.  Resonance searches focusing on
this decay channel have been presented in~\cite{Aaboud:2018xuw}, and
combined searches that consider the $bW$, $tZ$ and $th$ channels
already put strong constraints on such vectorlike
resonances~\cite{Aaboud:2018pii, Sirunyan:2018omb}.

In this work, we point out that the non-resonant ${t\bar t}hh$ process
is also of considerable interest in this context.  Due to the already
existing strong bounds on the vectorlike resonances, the non-resonant
production very often accounts for a large fraction of the total
${t\bar t}hh$ cross-section.  Furthermore, it carries distinct
information about the compositeness nature of the Higgs boson, as
opposed to the indirect information connected to the existence of the
heavy fermion resonances.  We also point out that the non-resonant
${t\bar t}hh$ process is closely connected to the ${\bar t}th$
process, but would be expected to display larger deviations from the
SM prediction.  We present here a first step in the analysis of such
processes in the context of the 14~TeV High Luminosity phase of the
LHC, and the possible High Energy LHC upgrade at 27~TeV. We explore
these questions in the framework of the ``Minimal Composite Higgs
Models"~\cite{Agashe:2004rs}, considering two possible
realizations of the sector of fermionic resonances.  Further details
can be found in the companion paper~\cite{MCHMtthh}.

%%%%%%%%%%%%%%%%%%%%%%%%%%%%%%%%%%%%%
\subsubsection{Theoretical Framework}
\label{MCHM}
%%%%%%%%%%%%%%%%%%%%%%%%%%%%%%%%%%%%%

The modern composite Higgs paradigm posits a new strongly coupled
sector that spontaneously breaks a global symmetry ${\cal G}$ down to
a subgroup ${\cal H}$, thus generating a number of
pseudo-Nambu-Goldstone bosons, some of which are identified
with the Higgs doublet.  The SM gauge group is contained in ${\cal H}$
and electroweak symmetry breaking takes place only when the Higgs
doublet acquires a vacuum expectation value, a process that is
dynamical and often calculable in such scenarios.  In addition to a
set of strong resonances of various spins, there is an ``elementary"
sector that mimics in its gauge quantum numbers the Standard Model.  The physical states are linear superpositions of the composite
and elementary states, realizing the paradigm of partial
compositeness, first proposed in~\cite{KAPLAN1991259}.  We refer the
reader to the full review in~\cite{Panico:2015jxa} for complete
details.  We restrict ourselves here to the MCHM scenarios, where
${\cal G} = SO(5)$ and ${\cal H} = SO(4)$.

In this work, we focus on the resonances associated to the top sector,
as these are the most relevant to the processes we study. Note section~\ref{sect-illus} for a complementary study on Higgs coupling to gauge bosons.  We
summarize here the essential features that enter into the
phenomenological analysis, referring the reader to the companion
work~\cite{MCHMtthh} for precise conventions and further details.
Since the fermionic sector is model dependent, we adopt two concrete
realizations.  Both share an elementary sector denoted by $q_L$ and
$t_R$, described by
%
\bea
\mathcal{L}_{\rm elem} &= \bar{q}_L i \slashed{D} q_L +\bar{t}_R i \slashed{D} t_R~,
\label{elem}
\eea
%
where $D$ stands for the $SU(3)_C \times SU(2)_L \times U(1)_Y$
covariant derivative.  $q_L$ and $t_R$ transform as $(3,2,1/6)$ and
$(3,1,2/3)$ under the SM gauge group.

\medskip
\noindent
\textit{\small \bf The MCHM$_5$}
\medskip

In this ``minimal" extension, one considers fermion resonances in a
${\bf 5}$ of $SO(5)$, which splits into a $SO(4)$ 4-plet,
$\Psi_4$, and a $SO(4)$ singlet, $\Psi_1$:
%
\be
\Psi_4 ~\sim~ (X_{5/3}, X_{2/3}, T, B)~;~   \Psi_1 ~\sim~ \tilde{T}~.
\label{comp5content}
\ee
%

 The states $(X_{5/3}, X_{2/3})$ transform as a $SU(2)_L$ doublet with
 exotic hypercharge $Y = 7/6$.  The states $(T, B)$ also transform as
 a $SU(2)_L$ doublet and have hypercharge $Y = 1/6$, while $\tilde{T}$
 is a $SU(2)_L$ singlet with hypercharge $Y = 2/3$.  The composite
 sector is described by
%
\bea
\mathcal{L}^{\mathbf{5}}_{\rm comp} &=&
 \bar{\Psi}_{\mathbf{4}} i \slashed{D} \Psi_{\mathbf{4}}-M_4 \bar{\Psi}_{\mathbf{4}} \Psi_{\mathbf{4}}+ \bar{\Psi}_{\mathbf{1}} i \slashed{D} \Psi_{\mathbf{1}}-M_1 \bar{\Psi}_{\mathbf{1}} \Psi_{\mathbf{1}}~.
\label{comp5}
\eea
%
There is also the possibility of mixing between the composite and
elementary states, described here by~\footnote{In principle, one can
choose different Yukawa couplings for the terms involving the ${\bf 4}$-plet, $\Psi_4$, and the singlet, $\Psi_1$.
See~\cite{MCHMtthh}.}
%
\bea
\mathcal{L}^{\mathbf{5}}_{\rm mix} &=&
y_L f \, \bar{q}_L^{\mathbf{5}} U \left[ \Psi_{\mathbf{4}} + \Psi_{\mathbf{1}} \right] 
+ y_R f \, \bar{t}_R^{\mathbf{5}} U \left[ \Psi_{\mathbf{4}} + \Psi_{\mathbf{1}} \right] + \mathrm{h.c.}
\label{Lmix5}
\eea
%
where $U$ parametrizes the Higgs field and $f$ is the ``Higgs decay
constant".  All the features required for our analysis follow from the
charge 2/3 fermion mass matrix, given in the $\{{\bar t}_L,~{\bar
T}_L,~{\bar X}_{2/3, L},~\overline{\tilde{T}}_L\}$ vs
$\{t_R,~T_R,~X_{2/3, R},~\tilde{T}_R\}$ basis by
%
\begin{align}
\mathcal{M}_{2/3}^{\mathbf{5}} =
\left[\begin{array}{cccc}
        0 & \frac{1}{2} y_{L} f (1+\sqrt{1-\xi}) & \frac{1}{2} y_{L} f (1-\sqrt{1-\xi}) & \frac{1}{\sqrt{2}} y_{L} f \sqrt{\xi} \\
        -\frac{1}{\sqrt{2}} y_{R} f \sqrt{\xi} & -M_4 & 0 & 0 \\
        \frac{1}{\sqrt{2}} y_{R} f \sqrt{\xi} & 0 & -M_4 & 0 \\
        y_{R} f \sqrt{1-\xi} & 0 & 0 & -M_1
      \end{array}\right]~,
\label{e:mass5}
\end{align}
%
where $\xi = v^2/f^2$ characterizes the deviations from a SM Higgs due
to compositeness (here $v = 246~{\rm GeV}$).  Consistency with current
Higgs measurements results in $\xi\lesssim 0.1$, or $f \gtrsim 800$
GeV~\cite{Falkowski:2013dza, Carena:2014ria, Buchalla:2014eca,
Sanz:2017tco, Liu:2017dsz, Banerjee:2017wmg, deBlas:2018tjm}.  The
remaining resonances have masses $M_{X_{5/3}} = M_4$ and $M_B =
\sqrt{M_4^2 + y_{L}^2 f^2}$.

\medskip
\noindent
\textit{\small \bf The MCHM$_{14}$}
\medskip

In the second scenario, the composite states span a ${\bf 14}$ of
$SO(5)$~\cite{Pomarol:2012qf, Panico:2012uw, Montull:2013mla, 
Carena:2014ria, Carmona:2014iwa, Kanemura:2016tan, Gavela:2016vte, 
Liu:2017dsz}. Under $SO(4)$, in addition to a 4-plet and a singlet, as in
Eq.~(\ref{comp5content}), we have an additional $SO(4)$ nonet:
%
\bea
\Psi_{\bf 9} &\sim& (U_{8/3}, U_{5/3}, U_{2/3}, Y_{5/3}, Y_{2/3}, Y_{-1/3}, Z_{2/3}, Z_{-1/3}, Z_{-4/3})~.
\eea
%
The $U$'s, $Y$'s and $Z$'s transform as $SU(2)_L$ triplets, with
hypercharges $Y = 5/3$, $2/3$ and $-1/3$, respectively.  The composite
Lagrangian given in Eq.~(\ref{comp5}) is supplemented by terms
involving $\Psi_9$:
%
\bea
\mathcal{L}^{\mathbf{14}}_{\rm comp} &=& \mathcal{L}^{\mathbf{5}}_{\rm comp} +
\mathrm{Tr}\left[\bar{\Psi}_{\mathbf{9}} i \slashed{D}\Psi_{\mathbf{9}} \right]-M_9 \mathrm{Tr}\left[\bar{\Psi}_{\mathbf{9}} \Psi_{\mathbf{9}}\right]~,
\eea
%
while the mixing with the elementary states is given by~\footnote{Here
$Q_L^{\bf 14}$ and $T_R^{\bf 14}$ are convenient extensions of $q_L$
and $t_R$ that simplify the writing of the Lagrangian.
See~\cite{MCHMtthh}.}
%
\bea
\mathcal{L}^{\mathbf{14}}_{\rm mix} &=&
y_L f \, {\rm Tr} \left[ U^\mathsf{T} \bar{Q}_L^{\mathbf{14}} U \Psi_{\mathbf{14}} \right]
+ y_R f \, {\rm Tr} \left[ U^\mathsf{T} \bar{T}_R^{\mathbf{14}} U
\Psi_{\mathbf{14}} \right] + \mathrm{h.c.}
\label{Lmix14}
\eea
%
where $\Psi_{\bf 14} \equiv \Psi_{\mathbf{9}} + \Psi_{\mathbf{4}} + \Psi_{\mathbf{1}}$.

The charge 2/3 mass matrix, in the $\{{\bar t}_L,~{\bar T}_L,~{\bar
X}_{2/3, L},~\overline{\tilde{T}}_L,~{\bar U}_{2/3, L},~{\bar Y}_{2/3,
L},~{\bar Z}_{2/3, L}\}$ vs 
$\{t_R,~T_R,$ $X_{2/3, R},~\tilde{T}_R,~U_{2/3, R},~Y_{2/3, R},~Z_{2/3, R}\}$ basis, is given by:
%
{ \small
\begin{align}
&\mathcal{M}_{2/3}^{\mathbf{14}} =
&\left[\begin{array}{ccccccc}
        0 & \frac{1}{2} y_{L} f a_+ & -\frac{1}{2} y_{L} f a_- & -\frac{\sqrt{5}}{4} y_{L} f s_{2h} & -\frac{1}{2} y_{L} f b_- &-\frac{1}{2} y_{L} f s_{2h} &\frac{1}{4} y_{L}f b_+ \\
        \frac{\sqrt{5}}{4} y_{R} f s_{2h} & -M_4 & 0 & 0&0&0& 0\\
        -\frac{\sqrt{5}}{4} y_{R} f s_{2h} &0& -M_4  & 0&0&0& 0\\
        y_{R}f\left(1-\frac{5}{4} s^2_h \right) & 0 & 0 & -M_1&0&0&0 \\
        \frac{\sqrt{5}}{4} y_{R} f s^2_h & 0 & 0 &0 & -M_9&0&0 \\
        -\frac{\sqrt{5}}{4} y_{R} f s^2_h &0 &0 &0 &0&-M_9&0 \\
        \frac{\sqrt{5}}{4} y_{R} f s^2_h & 0 & 0 &0 & 0&0&-M_9
      \end{array}\right]~,
\label{e:mass14}
\end{align}
}
%
where
$
s^2_{h} = \xi~,
\hspace{1em}
s_{2h} = 2\sqrt{\xi}\sqrt{1-\xi}~,
\hspace{1em}
a_\pm = 1 \pm \sqrt{1-\xi} - 2\xi~,
\hspace{1em}
b_\pm = \sqrt{\xi} \left(1 \pm \sqrt{1-\xi} \right)
$.

We give the charge $-1/3$ mass matrix in Ref.~\cite{MCHMtthh}, as it
plays a subdominant role.  The remaining states have masses
$M_{X_{5/3}} = M_4$, $M_{U_{8/3}} = M_{U_{5/3}} = M_{Y_{5/3}} =
M_{Z_{-4/3}} = M_9$.

An important distinction between the two scenarios is that when the
mixing is dominated by the nonet, the leading order operator coupling
the top quark to the Higgs doublet is the non-renormalizable operator
${\bar q}_L \tilde{H} t_R H^\dagger H$.  In contrast, mixing through a
4-plet or singlet lead to the SM operator ${\bar q}_L \tilde{H} t_R$
(plus corrections that are higher order in $H/f$).  In the former case
the ratio of the top Yukawa coupling to the top mass is three times
larger than in the second case.  Cases where the nonet plays a
comparable role to the 4-plet or singlet can then lead to interesting
enhancements in the top Yukawa coupling, which are not present in the
MCHM$_5$.

\medskip
\noindent
\textit{\small \bf Higgs Decays}
\medskip

The scenarios under consideration can also affect the Higgs decays,
and thus require a specification of how the remaining (light) fermions
are treated.  For concreteness, we assume that the RH bottom is
associated with resonances in a ${\bf 10}$ of
$SO(5)$~\cite{Carena:2014ria}.  For the remaining fermions, we choose
to replicate the scheme employed for the third family.  We also assume
that the lepton sector follows the same scheme as the quark sector.
Furthermore, we assume that the mixing angles between the elementary
and composite states associated with the light families are small, as
in ``anarchy" models of flavor.

Under the previous assumptions, one can express the partial widths as
a rescaling of the corresponding SM widths.  For the MCHM$_5$, one
finds~\cite{Carena:2014ria}
%
\bea
\Gamma(h \to b {\bar b}) &=& F_2(\xi)^2 \, \Gamma_{\rm SM}(h \to b {\bar b})~,
\nonumber \\ [0.5em]
\Gamma(h \to c {\bar c}) &=& F_1(\xi)^2 \, \Gamma_{\rm SM}(h \to c {\bar c})~,
\nonumber \\ [0.5em]
\Gamma(h \to \tau^+ \tau^-) &=& F_2(\xi)^2 \, \Gamma_{\rm SM}(h \to \tau^+ \tau^-)~,
\\ [0.5em]
\Gamma(h \to VV) &=& F_2(\xi)^2 \, \Gamma_{\rm SM}(h \to VV)~,
\nonumber \\ [0.5em]
\Gamma(h \to gg) &=& F_1(\xi)^2 \, \Gamma_{\rm SM}(h \to gg)~,
\nonumber
\eea
where
\be
F_1(\xi) ~=~ \frac{1-2\xi}{\sqrt{1-\xi}}~,
\hspace{1cm}
F_2(\xi) ~=~ \sqrt{1-\xi}~.
\label{Ffunctions}
\ee
%

For the MCHM$_{14}$, the bottom channel is controlled by $F_1$ instead
of $F_2$, and the $ggh$ coupling is controlled by
%
\bea
c_g^{\bf 14} &\approx&
\frac{
4 \left(1 - r_1\right) r_9
- \left(9 r_1 + 23 r_9 - 32 r_1 r_9\right) \xi
+ 4 \left(3 r_1 + 5 r_9 - 8 r_1 r_9\right) \xi ^2
}
{\sqrt{1-\xi }
\left[4 \left(1 - r_1 \right) r_9 - \left(3 r_1 + 5 r_9 - 8 r_1 r_9\right) \xi \right]}~,
\eea
%
where $r_1 = M_1 / M_4$ and $r_9 = M_9 / M_4$, instead of $F_1(\xi)$.

The total Higgs width in the MCHM models under consideration can then
be written as
%
\bea
\Gamma_5(h) &=& \left\{ F_2(\xi)^2 \left[ {\rm BR_{\rm SM}(b{\bar b})} + {\rm BR_{\rm SM}(VV)} + {\rm BR_{\rm SM}(\tau^+ \tau^-)} \right] \right.
\nonumber \\
&& \mbox{} + \left. F_1(\xi)^2 \left[ {\rm BR_{\rm SM}(gg)} + {\rm BR_{\rm SM}(c{\bar c})} \right] \right\} \Gamma_{\rm SM}(h)~,
\\ [0.5em]
\Gamma_{14}(h) &=& \left\{ F_2(\xi)^2 \left[{\rm BR_{\rm SM}(VV)} + {\rm BR_{\rm SM}(\tau^+ \tau^-)} \right] \right.
\nonumber \\
&& \mbox{} + \left. F_1(\xi)^2 \left[ {\rm BR_{\rm SM}(b{\bar b})} + {\rm BR_{\rm SM}(c{\bar c})} \right]
+ (c_g^{\bf 14})^2 \, {\rm BR_{\rm SM}(gg)} \right\} \Gamma_{\rm SM}(h)~,
\eea
%
and the branching fractions can also be written in terms of the
functions $F_1$, $F_2$, $c_g^{\bf 14}$, and SM quantities.  They
%These branching fractions 
are all that is needed to take into account the
effects of compositeness in Higgs decays.
%%%%%%%%%%%%%%%%%%%%%%%%%%%%%%%%%%%%%%%%%%%%%%%%
\subsubsection{Parameter Space and Results}
\label{analysis}
%%%%%%%%%%%%%%%%%%%%%%%%%%%%%%%%%%%%%%%%%%%%%%%%
The parameters of the models are $M_i$, $f$, $y_L$ and $y_R$.  Taking
into account the freedom to absorb phases through field redefinitions,
one can see that there is only one physical phase in the MCHM$_5$,
while there are two phases in the MCHM$_{14}$.  For simplicity, we
will take all parameters to be real, so that the free parameters can be
taken as follows:
\begin{itemize}
\item MCHM$_5$: $f$, $|M_1|$, $|M_4|$, ${\rm sign}(M_1)$, $y_L$ and $y_R$.
\item MCHM$_{14}$: $f$, $|M_1|$, $|M_4|$, $|M_9|$, ${\rm sign}(M_1)$, ${\rm sign}(M_4)$, $y_L$ and $y_R$.
\end{itemize}
One of these parameters can be further fixed by requiring that the top
mass be reproduced.  We choose to fix $y_R$ in this way.  We use here
the running top mass at the scale of the resonances, which will be
typically around $2-3$~TeV. We take $\bar{m}_t = 150$~GeV. For
$t\bar{t}h$, and also for the parts of $t\bar{t}hh$ to be described in
the following section, the relevant scales are of the order of a
couple hundred GeV. We therefore distinguish between the high-scale
running top mass (relevant for the diagonalization of the mass
matrix), and a low scale running top mass, relevant to the physical
processes of interest.  We take for the latter the pole top mass of
$m_t = 173$~GeV, which also enters in kinematical quantities.  To
first approximation, this takes into account the running between the
two scales.  We take the Higgs mass as an independent parameter,
referring the reader to~\cite{MCHMtthh} for further discussion on this
point.

For the MCHM$_5$, we consider the following ranges for the parameters:
\begin{align*}
|M_1|& \in [800, 3000]~{\rm GeV}~, &  M_4 &\in [1200, 3000]~{\rm GeV}~,  \\
f&\in [800, 2000]~{\rm GeV}~,          &  y_L&\in [0.5, 3] ~.
\end{align*}

For the MCHM$_{14}$, we use:
\begin{align*}
|M_1|& \in [800, 3000]~{\rm GeV}~, &  |M_4| &\in [1200, 3000]~{\rm GeV}~,  &  M_9 &\in [1300, 4000]~{\rm GeV}~,  \\
f&\in [800, 2000]~{\rm GeV}~,          &  y_L&\in [0.5, 3] ~.
\end{align*}
We take an upper limit on $y_L < 3$, in order to remain in the (semi-)
perturbative regime and thus justify the present tree-level analysis.
For the same reason, we also check that $y_R$, as determined by the
top mass, is below 4.
%
\begin{figure}[t]
\centering
\includegraphics[width=0.475\textwidth]{\main/section9/plots/MCHM5_gridplot.pdf}
\hspace{3mm}
\includegraphics[width=0.485\textwidth]{\main/section9/plots/MCHM14_GridPlot_f_1200_M9_2TeV.pdf}
\caption{Display of the values of the normalized top Yukawa coupling,
$y_{\rm top} / y_{\rm top}^{\rm SM}$, in the $M_1$-$M_4$ plane.  Blue
colors indicate a suppression and red colors an enhancement.  Also
shown the curves of constant $M_{T^{(1)}}$, the mass of the lightest $Q=2/3$
vectorlike resonance.  The darker bands indicate the approximate
current direct exclusion of top partner VLQ resonances, assuming decays into
$bW$, $tZ$ and $th$ ~\cite{Aaboud:2018pii, Sirunyan:2018omb}.}
\label{fig:ytvsM1M4}
\end{figure}
%
In Fig.~\ref{fig:ytvsM1M4} we show the normalized top Yukawa coupling,
$y_{\rm top} / y_{\rm top}^{\rm SM}$, in the $M_1$-$M_4$ plane for
both the MCHM$_5$ and MCHM$_{14}$ scenarios.  We fix $y_L = 2$ and $f
= 1200$~GeV, and $M_9 = 2$~TeV for the MCHM$_{14}$.  In the MCHM$_5$,
the scaling with $f$ is, to first approximation, given by the function
$F_1(\xi)$ in Eq.~(\ref{Ffunctions}), while for the MCHM$_{14}$ it is
intertwined with the other parameters in a more complicated way.  We
see that the MCHM$_5$ always displays a suppression of the top Yukawa
coupling compared to the SM limit, while the MCHM$_{14}$ can display
an enhancement in certain regions of parameter space, as pointed out
in~\cite{Liu:2017dsz}.  We also show in the figure, curves of constant
$M_{T^{(1)}}$ (red lines) and the approximate direct exclusion region
(dark bands).  The white area corresponds to the region in parameter
space where it is not possible to reproduce the top quark mass.  We
also show the region where the $ggh$ coupling deviates by more than
20\% from unity, as this region is expected to be in tension with the
current constraints on Higgs couplings~\cite{Khachatryan:2016vau}.
%%%%%%%%%%%%%%%%%%%%%%%%%%%%%%%%%%%%%%%%%%%%%%%%
\subsubsubsection{Implementation into Event Generator}
\label{mg}
%%%%%%%%%%%%%%%%%%%%%%%%%%%%%%%%%%%%%%%%%%%%%%%%
We implement both models in FeynRules
(v2.3)~\cite{Alloul:2013bka} and produce an associated UFO file for
each model, that can be interfaced with MadGraph 5
(v2.6.2)~\cite{Alwall:2014hca}.  The numerical input from the
diagonalization of the mass matrices is then fed via a custom-written
Python script into the param\_card.dat for processing within MG5.  We
simulate the ${t\bar t} h$ and ${t\bar t} hh$ processes in MG5.  The 
output is then ready to be run through PYTHIA~8.2~\cite{Sjostrand:2014zea}, which
takes care of the decays of the top quarks and the Higgs.  We code the
Higgs branching fractions, as described above,
into Pythia.  We also check that the deviations from the SM in
the top quark properties are negligible, since the new physics is
rather heavy.  The output is then ready to pass through fast 
simulation~\cite{deFavereau:2013fsa} or the full simulation of the LHC experiments for HL-LHC.

%%%%%%%%%%%%%%%%%%%%%%%%%%%%%%%%%%%%%%%%%%%%%%%%
\subsubsubsection{The $t\bar{t}h$ Process}
\label{tth}
%%%%%%%%%%%%%%%%%%%%%%%%%%%%%%%%%%%%%%%%%%%%%%%%
To an excellent approximation, the $t\bar{t} h$ process in the MCHM
is related to the corresponding SM process by a simple rescaling:
%
\be
\sigma_{\rm MCHM}(t\bar{t}h) ~=~ \left( \frac{y_t}{y_t^{\rm SM}} \right)^2 \, \sigma_{\rm SM}(t\bar{t}h)~.
\label{sigmatth}
\ee
%
All the modifications due to Higgs compositeness, or mixing with
vector-like fermions, enter only through the top Yukawa coupling.
Therefore, only a modification in the total rate is expected, but not
in kinematic distributions.

%%%%%%%%%%%%%%%%%%%%%%%%%%%%%%%%%%%%%%%%%%%%%%%%
\subsubsubsection{The $t\bar{t}hh$ Process}
\label{tthh}
%%%%%%%%%%%%%%%%%%%%%%%%%%%%%%%%%%%%%%%%%%%%%%%%
For the $t\bar{t}hh$ process there are two qualitatively different
contributions:
%
\begin{enumerate}
\item Resonant processes, involving the production and decay (in the
th channel) of heavy vector-like states of charge 2/3 (top partners).
\item Non-resonant processes: these are defined by the diagrams that
do not involve the production of the vector-like resonances.
\end{enumerate}
%
\begin{figure}[t]
\centering
\includegraphics[width=0.48\textwidth]{\main/section9/plots/MCHM5-1-14.pdf}
\put(-110,35){\footnotesize14 TeV}
\hspace{3mm}
\includegraphics[width=0.48\textwidth]{\main/section9/plots/MCHM5-1-27.pdf}
\put(-110,35){\footnotesize 27 TeV} 
\caption{Distribution of the invariant mass of the top quark and the
hardest Higgs boson in the MCHM$_5$ ($M_{1} = -2500$~TeV, $M_4 =
2$~TeV, $f=1.8$~TeV, $y_L=1$).  The blue histogram shows the
distribution of the full $t\bar{t}hh$ process in the MCHM$_5$, while
the NR-tthh cross-section is shown in red.  For comparison, we also
show in green the SM $t\bar{t}hh$ distribution.  Plots generated with
MadAnalysis~5~\cite{Conte:2012fm}.}
\label{fig:htdist}
\end{figure}
%
The presence of resonant processes can lead to important enhancements
in the $t\bar{t}hh$ cross-section w.r.t.~the SM, depending on their
mass.  The non-resonant process carries information that is distinct
from the resonant part.  It is therefore useful to
define a ``non-resonant cross-section" as obtained from this subset of
diagrams, which we label as ``NR-tthh".  One can similarly define a
resonant cross section in terms of the diagrams involving QCD
vector-like pair production.  We find that, to an excellent
approximation, the total $t\bar{t}hh$ cross-section is given by the
sum of these two cross-sections. 

In Fig.~\ref{fig:htdist} we show the $ht$ invariant mass distribution
for the resonant and non-resonant processes for a particular point in
the MCHM$_5$.  For comparison, we also show the SM $t\bar{t}hh$
cross-section.  We see that the NR-tthh follows the SM cross-section,
but displays a suppression.  We also see that the relative importance
of the resonant process w.r.t.~the non-resonant one increases with
larger c.m. energies.  The cross-section for both processes also
increases significantly with the c.m. energy (by a factor of 7 in the
total $t\bar{t}hh$ cross-section when going from 14 to 27~TeV, and by
a factor of 5 when restricted to NR-tthh). 

%%%%%%%%%%%%%%%%%%%%%%%%%%%%%%%%%%%%%%%%%%%%%%%%
\subsubsubsection{The Non-Resonant $t\bar{t}hh$ Process}
\label{NRtth}
%%%%%%%%%%%%%%%%%%%%%%%%%%%%%%%%%%%%%%%%%%%%%%%%
The diagrams in the MCHM scenarios contributing to the NR-tthh process
fall into three categories:
%
\begin{enumerate}
\item Those that involve only the tth vertex.
\item Those that involve the trilinear Higgs self-interaction~\ref{sec3}:
$\lambda = \left[ (1 - 2 \xi)/\sqrt{1 - \xi} \right] \lambda_{\rm
SM}$.
\item Those that involve the tthh vertex (``double Higgs" Yukawa vertex).
\end{enumerate}
%
The first two categories correspond to sets of \textit{diagrams} that
are identical to those in the SM. The third type involves diagrams
that have no counterpart in the SM~\cite{Contino:2012xk}.  The latter
is closely connected to the Higgs compositeness aspect of the MCHM
scenarios, and it would therefore be extremely interesting if one
could get information about such effects experimentally.

In order to get a sense for the relative importance of the different
physical subprocesses, we simulate the NR-tthh cross section
turning off, in turn, the double Higgs Yukawa coupling and the trilinear
coupling.  We find that the effects of the double Higgs Yukawa coupling are
typically at the couple to few percent level in MCHM$_5$ and MCHM$_{14}$ if the $t\bar{t}h$ signal strength,
$\mu(t\bar{t}h) \equiv \sigma(t\bar{t}h)/{\sigma(t\bar{t}h)}_{SM} < 1$, 
and at most $2\%$ in MCHM$_{14}$ if $\mu(t\bar{t}h) > 1$, with a mild dependence
on the c.m. energy (at~14 and 27~TeV) in all cases.
We also find that the effect of the trilinear Higgs self-interaction
can be around $15\%$ in MCHM$_5$ and MCHM$_{14}$ if $\mu(t\bar{t}h) < 1$, and $10\%$ in MCHM$_{14}$ if $\mu(t\bar{t}h) > 1$ at a c.m. energy of 14~TeV, decreasing to a few
percent at higher c.m. energies in all cases.  For comparison, the effect of the
trilinear Higgs self-interaction in the SM $t\bar{t}hh$ cross-section
is about $20\%$, with a very mild c.m. energy dependence.  Thus, the
NR-tthh (like the SM $t\bar{t}hh$ cross-section) is largely determined
by the top Yukawa interaction, and the two are, to a first
approximation, related by a scaling factor $(y_t / y_t^ {\rm SM})^4$.
This explains the result seen in Fig.~\ref{fig:htdist}, with the
suppression arising from the suppression of the top Yukawa coupling in
the MCHM$_5$.


The previous observation also leads to a strong correlation between the $t\bar{t}h$ and the NR-tthh processes, as shown in Fig.~\ref{fig:nrtthhvstth}. Due to the different scaling with the top Yukawa coupling, the deviations from the SM in the NR-tthh process are larger than those in $t\bar{t}h$.
%%%%%%%%%%%%%%%%%%%%%%%%%%%%%%%%%%%%%%%%%%%%%%%%
%\subsubsubsection{Sample Points}%
\subsubsubsection{Set of Example Points}
\label{benchmarks}
%%%%%%%%%%%%%%%%%%%%%%%%%%%%%%%%%%%%%%%%%%%%%%%%
We show in Table \ref{fig:benchmarkTable1} 
%and \ref{fig:benchmarkTable2}, 
a number of points selected as examples that illustrate, in
more detail, the properties of the MCHM$_5$ and MCHM$_{14}$.  
%They are indicated in% 
These properties are reflected in Figs.~\ref{fig:nrtthhvstth}, ~\ref{fig:tthvsf} and~\ref{fig:tthhvsMT4}, where these points are indicated. The MCHM$_5$ points are labelled as P$_i$, i=1 to 5, and MCHM$_{14}$ points as P'$_j$, with j=1 to 4.
%
\begin{table}[t]
\centering
\includegraphics[width=0.85\textwidth]{\main/section9/plots/ExamplePoints_MCHM5_14.pdf}
\caption{Sample points for MCHM$_5$ with M$_1$ M$_4$ same sign and opposite sign and for MCHM$_{14}$ with M$_1$ and M$_4$ both $<0$ and $\mu({\rm tth})>1$.}
\label{fig:benchmarkTable1}
\end{table}
%
\begin{figure}[t]
\centering
\includegraphics[width=0.45\textwidth]{\main/section9/plots/MCHM5_XStth_Norm_XStthh_NR_Norm-14_and_27.pdf}
\hspace{1cm}
\includegraphics[width=0.45\textwidth]{\main/section9/plots/MCHM14_XStth_Norm_XStthh_NR_Norm-14_and_27.pdf}
\caption{Correlation between the $t\bar{t}h$ and
non-resonant $t\bar{t}hh$ signal strengths ($\mu$), for 14 and 27~TeV c.m.~energies. The left (right) plots correspond to the MCHM$_5$ (MCHM$_{14}$)}
\label{fig:nrtthhvstth}
\end{figure}
%
The points for the MCHM$_5$ exhibit a suppression in $\mu(t\bar{t}h)$ that ranges from about 15\% (roughly at the current
95\% C.L.~limit~\cite{Aaboud:2018urx, Sirunyan:2018hoz}) to a few
percent, a sensitivity that might be achievable by the end of the HL
phase of the LHC run (Fig.~\ref{fig:tthvsf},a).  The smaller deviations from the SM are
associated with larger values of $f$ (Fig~\ref{fig:tthvsf},a).  The Table 19 and Fig~\ref{fig:tthhvsMT4} show that the
$t\bar{t}hh$ process can exhibit an enhancement if the fermion
resonances are light enough.  As expected, this enhancement increases
with increasing c.m.~energy. For the points 2, 4 and 5 in the MCHM$_{5}$, the resonant production is not enough to produce an
enhancement in $t\bar{t}hh$ compared to the SM, although these points correspond to two different cases; the resonances for Point 2 are slightly beyond the current direct limit whereas, on the contrary, much beyond that limit for points 4 and 5. In this case, the
$t\bar{t}hh$ process is easily dominated by the NR-tthh process, as
defined above.  For completeness, Table~\ref{fig:benchmarkTable1} includes the
spectrum of resonances, and the BRs for the lightest $Q=2/3$ one.  It
decays mostly into the standard $th$, $Wb$ and $tZ$ channels (with BRs
that are model dependent), but in some cases it has a non-negligible
non-standard BRs, such as into the $W^+W^- t$ channel.

The set of example points for MCHM$_{14}$ in Table~\ref{fig:benchmarkTable1} exhibits an enhancement of the top Yukawa
coupling, due to the effect described in section~\ref{MCHM} and reflected in Fig~\ref{fig:tthvsf},b.  These
enhancements can easily be of the order of 10-20\%.  Interestingly,
Point 1 shows that the enhancement can be as large as 40\% (while
being consistent with a sufficiently small deviation in the $ggh$
vertex~\cite{MCHMtthh}).  The four points display as well, an enhancement
in the $t\bar{t}hh$ process.  While about half of the rate is due to
resonant production in Points 1 and 2, for points 3 and 4 the
enhancement arises dominantly from the non-resonant process,
reflecting the enhancement in the top Yukawa coupling.  In Table~\ref{fig:benchmarkTable1} are displayed,
 the spectrum of the 5 resonances in the MCHM$_{5}$ and of the 3 lightest 2/3 resonances, the lightest B resonance and the lighest 5/3 resonance out of the total of 14 resonances of the MCHM$_{14}$.  
 All the selected points for MCHM$_{14}$ lie in the
$M_1 < 0$, $M_4 < 0$ quadrant of the right panel of
Fig.~\ref{fig:ytvsM1M4}. The properties of the other quadrants are
qualitatively rather similar to those of the MCHM$_5$ (see~\cite{MCHMtthh}).
\begin{figure}[!htb]
\centering
\includegraphics[width=0.45\textwidth]{\main/section9/plots/MCHM5_f_XStth_Norm_MT4_Q2.pdf}
\put(-35,45){a)}
\hspace{1cm}
\includegraphics[width=0.47\textwidth,scale=1.2]{\main/section9/plots/MCHM14_f_XStth_Norm_MT4_Q3.pdf}
\put(-45,45){b)}
\caption{The $t\bar{t}h$ signal  strength as a function of the $f$-scale,
for 14 and 27~TeV c.m. energies, with color coded the lightest
vector-like mass.  The left (right) plots correspond to Q2 of MCHM$_5$ (Q3 of MCHM$_{14}$). The blue arrow indicates that the point P4 is outside the horizontal range of the plot with f=2450 GeV.}
\label{fig:tthvsf}
\end{figure}
\begin{figure}[!htb]
\centering
\includegraphics[width=0.45\textwidth]{\main/section9/plots/MCHM5_MT4_XStthh_Norm-14_and_27.pdf}
\put(-60,191){a)}
\hspace{1cm}
\includegraphics[width=0.45\textwidth]{\main/section9/plots/MCHM5_MT4_XStthh_NR_over_XStthh-14_and_27.pdf}
\put(-60,45){b)}
\caption{The left plot shows the $t\bar{t}hh$ signal  strength as a function of the
lightest $Q = 2/3$ vector-like mass, T$^{(1)}$ for 14 and 27~TeV c.m. energies for the MCHM$_5$. The right plot shows the ratio between the non-resonant $t\bar{t}hh$ cross section and the total $t\bar{t}hh$ cross section as a function of T$^{(1)}$
% the lightest $Q = 2/3$ vector-like mass,%
for 14 and 27~TeV c.m. energies for the MCHM$_5$.}
\label{fig:tthhvsMT4}
\end{figure}
%
%%%%%%%%%%%%%%%%%%%%%%%%%%%%%%%%%%%%%%%%%%%%%%%%
\subsubsection{Experimental perspectives}
\label{perpectives}
%%%%%%%%%%%%%%%%%%%%%%%%%%%%%%%%%%%%%%%%%%%%%%%%
%
A deviation from the SM in the $t{\bar t}h$ production is an essential measurement for MCHM. An increase will reject the MCHM$_5$ scenario and greatly refine the areas of the parameter space where MCHM$_{14}$ would be valid. A deficit instead, would make MCHM$_5$ and MCHM$_{14}$ both possible. The measurement of this observable is expected to be achieved within 5\% accuracy at the HL-LHC (sections~\ref{sec:2:top},\ref{sec:2:exp_combination},\ref{sec:2:exp_kappa}) and thus with very high accuracy at HE-LHC. The $t{\bar t}hh$ production process plays a major role in MCHM searches. Deviations from the SM expectation (deficit or increase) can be significant in both MCHM scenarios. The $t{\bar t}hh$ production cross-section is around 1 fb (section~\ref{sec3.1}) at tree level whereas $t{\bar t}h$ is about 500 times larger (section~\ref{sec:2_HXSWG1}). Therefore the aim at HL-LHC will be to evidence this process and discover if a strong deviation from SM. For exploring MCHM, higher energy together with higher luminosity (HE-LHC) is really a plus if even not a need. 

\textbf{Acknowledgments:}
This work was supported by the S\~ao Paulo Research Foundation
(FAPESP) under Grants No.~2016/01343-7, No.~2013/01907-0,
No.~2015/26624-6 and No.~2018/11505-0 and by Science Without Borders/CAPES for UNESP-SPRACE under the Grant No.~88887.116917/2016-00.