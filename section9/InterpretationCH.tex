\begin{center}
 {L. Vecchi}
\end{center}
\label{sec9:CHM}

Composite Higgs (CH) models postulate that the Standard Model (SM) Higgs sector is UV-completed by a strongly-coupled dynamics characterized by some scale $m_*$ not too far above the TeV. Because by analogy with QCD $m_*$ can naturally be small compared to any existing microscopic scale, this framework provides an attractive solution to the hierarchy problem. 

Historically, precision {\emph{indirect}} tests, mainly from electroweak (EW) data, have resulted in important constraints on strongly-coupled extensions of the SM. The discovery of the Higgs boson has finally removed the uncertainty associated to the value of $m_h$ but otherwise has not improved those bounds qualitatively. On the other hand, {\emph{direct}} access to the Higgs boson properties has had a qualitative impact on CH scenarios: we now know that viable realizations must contain a light scalar resonance $h$ with properties that mimic those of the SM Higgs boson. This observation excludes solutions to the hierarchy problem realized via the so-called Higgless scenarios (like old-fashioned technicolor), but leaves open a number of options, a representative set of which will be discussed here. Overall, CH scenario with a Higgs-like resonance continue to offer a very compelling explanation of the weak scale.




In this section we will focus on two representative classes of CH scenarios that predict a light scalar with SM-like couplings:
\begin{itemize}
\item[1)] the Strongly-Interacting Light Higgs (SILH). In this class the exotic strong dynamics generates a light scalar doublet $H$ with the same $SU(2)_w\times U(1)_Y$ charges of the SM Higgs, and it is the latter who spontaneously breaks the EW symmetry~\cite{Giudice:2007fh}. The doublet $H$ may be part of a Nambu-Goldstone multiplet, or simply be an accidentally light scalar. The physical Higgs boson $h$ belonging to the composite doublet behaves as the SM Higgs boson up to corrections induced by higher-dimensional operators suppressed by the strong coupling scale $m_*$. 
\item[2)] the Strongly-Interacting Light Dilaton (SILD). In this class of theories the strong dynamics is assumed to feature the spontaneous breaking of an approximate scale invariance at a scale $f_D$. In such a framework the low energy effective field theory (EFT) possesses an approximate Nambu-Goldstone mode, the dilaton, which automatically has couplings aligned along the direction of those of the SM Higgs~\cite{Goldberger:2008zz}. The key difference compared to the SILH is that this is a non-decoupling scenario, in which the new physics threshold is controlled by the EW scale. %This makes it especially challenging to envision concrete strongly-coupled realizations of this scenario that are compatible with data.... 
{Similar considerations apply to all scenarios in which the Higgs-like particle $h$ is not embedded in an EW doublet $H$.} %, where the chiral Lagrangian approach is adopted.}
\end{itemize}
%The assumption that the UV description is strongly coupled allows us to push them as far away from $\sim m_t$ as possible, say at $m_*\lesssim{\cal O}(4\pi v)$. %This scenario is intended to mimic the signatures of an accidentally light composite scalar, or Higgs-impostor, that might appear....


%Besides a Higgs-like state (that we will denote by $h$ in both SILH and SILD scenarios), several realizations of the CH proposed in the literature predict the existence of additional particles that are potentially within the reach of the LHC. %Concrete examples are exotic scalars in non-minimal $G\to H$ cosets, fermion resonances (often postulated to reduce the fine-tuning), and vector resonances in Little Higgs models. Here we will assume that all such states are heavy enough to be directly inaccessible, or so broad that cannot be interpreted as resonances.~\footnote{For an explicit comparison between direct and indirect probes of the heavy scale $m_*$ see, e.g., \cite{Thamm:2015zwa}.} Within this assumption, an EFT formalism is particularly appropriate.


The main goal of this section is to (1) characterize the deviations in the couplings of the Higgs-like state $h$ from the SM expectation; (2) estimate the sensitivity of the HL and HE-LHC on the CH picture. We will focus on modifications of the on-shell couplings, as opposed to off-shell rates like double-Higgs production or $VV\to VV$ scattering. 









%%%%%%%%%%%%%%%%%%%%%
\begin{table}[t]
\caption{\small List of the dimension-6 operators relevant to our study of modified Higgs couplings in the SILH class. We use the basis of~\cite{Giudice:2007fh}. Here $y_\psi$ are the SM Yukawa couplings and $V=Z,W$. %CP-odd versions of ${\cal O}_{HW,HB,\gamma,g}$ can be obtained by replacing one field strength with the corresponding dual, $F^{\mu\nu}\to\frac{1}{2}\epsilon^{\mu\nu\alpha\beta}F_{\alpha\beta}$. The third column shows the dominant on-shell Higgs processes that the operator contributes to. %A star in the last column identifies the operators that are constrained by EW precision data.
\label{tab1}}
\begin{center}
%\resizebox{\textwidth}{!}
{
\begin{tabular}{c|c|c} 
\rule{0pt}{1.2em}%
Operator name & Operator definition & Main on-shell process \\
\hline
${\cal O}_H$ & $\frac{1}{2}~\partial_\mu(H^\dagger H)\partial^\mu(H^\dagger H)$ & $h\to\psi\bar\psi, VV^*$     \\
%${\cal O}_T$ & $\frac{1}{2}~(H^\dagger\overleftrightarrow{D_\mu}H)(H^\dagger\overleftrightarrow{D^\mu}H)$ & $h\to ZZ^*$ & $\star$ \\
${\cal O}_6$ & $\lambda_h(H^\dagger H)^3$ & $h^*\leftrightarrow hh$   \\
${\cal O}_y$ & $\sum_{\psi=u,d,e} y_\psi~\overline\psi_L H\psi_R(H^\dagger H)$ & $h\to\psi\bar\psi$   \\
\hline
%${\cal O}_W$ & $\frac{i}{2}~g(H^\dagger\sigma^i\overleftrightarrow{D_\mu} H)(D_\nu W^{\mu\nu})^i$ & $h\to VV^*,\gamma Z$ & $\star$\\
%${\cal O}_B$ & $\frac{i}{2}~g'(H^\dagger\overleftrightarrow{D_\mu} H)(\partial_\nu B^{\mu\nu})$ & $h\to ZZ^*,\gamma Z$ & $\star$ \\
%
${\cal O}_{HW}$ & $ig(D^\mu H)^\dagger\sigma^i(D^\nu H)W_{\mu\nu}^i$ & $h\to VV^*,\gamma Z$   \\
${\cal O}_{HB}$ & $ig'(D^\mu H)^\dagger(D^\nu H)B_{\mu\nu}$ & $h\to ZZ^*,\gamma Z$  \\
\hline
${\cal O}_{g}$ & $g_s^2H^\dagger H G^a_{\mu\nu}G^{a\mu\nu}$ & $h\to gg$   \\
${\cal O}_{\gamma}$ & $g'^2H^\dagger HB_{\mu\nu}B^{\mu\nu}$ & $h\to \gamma\gamma,\gamma Z,ZZ$  
%
\end{tabular}
}
\end{center}
\end{table}
%%%%%%%%%%%%%%%%%%%%%%%





\subsubsection{The SILH}

The operators of dimension 6 that dominantly impact on-shell processes involving $h$ are collected in table \ref{tab1} under the assumption that $H$ is an EW doublet. We do not include operators that are severely constrained by precision data, and for this reason are expected to lead to negligible corrections to the rates induced by those in the table.~\footnote{These are ${\cal O}_T=(H^\dagger\overleftrightarrow{D_\mu}H)(H^\dagger\overleftrightarrow{D^\mu}H)/2$, ${\cal O}_W={i}g(H^\dagger\sigma^i\overleftrightarrow{D_\mu} H)(D_\nu W^{\mu\nu})^i/2$, ${\cal O}_B=ig'(H^\dagger\overleftrightarrow{D_\mu} H)(\partial_\nu B^{\mu\nu})/2$, current-current interactions $H^\dagger \overleftrightarrow{D_\mu}H \bar \psi\gamma^\mu \psi$ and $H^\dagger \tau^a\overleftrightarrow{D_\mu}H \bar \psi\gamma^\mu \tau^a\psi$ containing non-universal couplings to the SM fermions $\psi$, and dipole operators. The operator ${\cal O}_T$ is constrained by the EW $\rho$ parameter; whereas ${\cal O}_W+{\cal O}_B$ by the EW $S$ parameter. Current-current operators are constrained by LEP and the non-observation of rare flavor-violating processes. Dipole operators are severely constrained by measurements of electric and magnetic moments. See, e.g.,~\cite{Contino:2013kra} for more details.%(The only exception is provided by those interactions involving the top.)
} In particular, in realistic SILH scenarios the Higgs coupling to fermions must be aligned to the SM Yukawas. For illustrative purposes, here we further simplify our discussion specializing on realizations in which the SM fermions are all coupled analogously to the strong sector, such that a universal fermionic operator ${\cal O}_y$ is sufficient. 

The observables that are mostly affected by the new interactions are shown in the third column of table \ref{tab1}. An estimate of the various Wilson coefficients in concrete CH models reveals that corrections to $h\to WW^*,ZZ,\psi\bar\psi$ are typically dominated by ${\cal O}_{H,y}$~\cite{Giudice:2007fh}. Those to the radiative processes $h\to gg,\gamma\gamma$ are controlled by 1-loop diagrams with an insertion of ${\cal O}_{H,y}$ if the doublet $H$ is a Nambu-Goldstone mode of the strong dynamics, but may also receive important contributions from ${\cal O}_{g,\gamma}$ if $H$ is an accidentally light resonance. For what concerns $h\to\gamma Z$ we find that ${\cal O}_{H,y}$ give contributions parametrically comparable to ${\cal O}_{HW, HB}$. The same is true for ${\cal O}_{\gamma}$, but only when $H$ is not a Nambu-Goldstone mode. Because the sensitivity on $h\to\gamma Z$ is appreciably weaker than $h\to\gamma\gamma$, it makes sense to simplify our discussion by neglecting the impact of ${\cal O}_{HW, HB}$ in our fit. Similarly, will ignore ${\cal O}_6$ since this operator only controls the very purely known Higgs self-couplings.

From these considerations follows that the leading on-shell signatures of the Higgs-like state $h$ in SILH scenarios can be characterized by the reduced set of operators 
\begin{eqnarray}\label{SILH}
\delta{\cal L}_{\rm SILH}=\frac{g_*^2}{m_*^2}\bar c_H{\cal O}_H+\frac{g_*^2}{m_*^2} \bar c_y{\cal O}_y+\frac{g_*^2}{16\pi^2m_*^2}\bar c_g{\cal O}_g+\frac{g_*^2}{16\pi^2m_*^2}\bar c_\gamma{\cal O}_\gamma,
\end{eqnarray}
where $\bar c_{H,y,g,\gamma}$ are expected to be of order unity and $g_*, m_*$ are the typical couplings and mass scale of the new physics. We included a factor of ${g_*^2}/{16\pi^2}$ in front of the last two operators in order to emphasize their radiative nature~\cite{Giudice:2007fh}. %Note that $\delta{\cal L}_{\rm SILH}$ is invariant under the SM gauge symmetry, since by assumption $H$ is the only source of EW symmetry breaking. 
We further assumed CP is approximately satisfied by the strong sector.

We can now match the Wilson coefficients appearing in \eqref{SILH} onto the phenomenological Lagrangian \eqref{eq:kappa.EFT.2} for the light boson $h$, up to additional interactions that are irrelevant to the present analysis. The resulting modified Higgs couplings are collected here:
\begin{eqnarray}\label{4parSILH}
c_V=1-\frac{\bar c_H}{2}\xi,~~~~~c_y=1-\left(\frac{\bar c_H}{2}+\bar c_y\right)\xi,~~~~~c_g=2 \bar c_g\xi,~~~~~c_\gamma=\bar c_\gamma\xi
\end{eqnarray}
where we defined 
\begin{eqnarray}\label{xi}
\xi\equiv\frac{g_*^2v^2}{m_*^2}\equiv\frac{v^2}{f^2},
\end{eqnarray}
with $v=246$ GeV.
%\ba\label{Lh}
%{\cal L}_h&=&\frac{1}{2}\partial_\mu h\partial^\mu h-\frac{1}{2}m_h^2h^2\\\no
%&+&\left[m_W^2W^+_\mu W^{-\mu}+\frac{1}{2}m_Z^2Z_\mu Z^\mu\right]\left(1+2c_V\frac{h}{v}+\cdots\right)\\\no
%&-&\sum_{\psi=u,d,l}m_\psi\bar\psi\psi\left(1+c_y\frac{h}{v}+\cdots\right)\\\no
%&+&\left[c_{g}G^a_{\mu\nu}G^{a\mu\nu}+c_{\gamma}\gamma_{\mu\nu}\gamma^{\mu\nu}%+c_{ZZ}Z_{\mu\nu}Z^{\mu\nu}+2c_{Z\gamma}Z_{\mu\nu}\gamma^{\mu\nu}
%\right]\frac{h}{v}\\\no
%&+&\cdots
%\ea
Note that our assumption of SM fermion-universality implies that $c_y$ is a single real number (independent of $\psi=u,d,e$). This leaves us with a total of 4 independent parameters~\eqref{4parSILH}. Our truncation of the EFT to dimension 6 operators is crucially associated to the working hypothesis $\xi\ll1$: operators of higher dimension are suppressed by larger powers of $\xi$. 
%The SM is indeed recovered decoupling the new physics scale $m_*$ ($\xi\to0$). 





\paragraph{Analysis}

The expected reach of the HL ($3$ ab$^{-1}$) and HE ($15$ ab$^{-1}$) LHC on the modified Higgs couplings has been presented in Section \ref{sec:kappavsEFT}. Here we specialize to scenarios defined by the 4 couplings $c_{V,y,g,\gamma}$. 
%\ba
%&&{\rm HL(S1)}\begin{cases}
%c_V=1\pm0.017\\
%c_y=1\pm0.023\\
%c_g=0\pm0.024\\
%c_\gamma=0\pm0.073
%\end{cases}~~~~~~~~~
%\rho=\left(
%\begin{matrix}
%1 & 0.65 & -0.34 & 0.15\\
%0.65 & 1 & -0.42 & -0.33\\
%-0.34 & -0.42 & 1 & 0.13\\
%0.15 & -0.33 & 0.13 & 1
%\end{matrix}
%\right)
%\\\no
%&&{\rm HE(S1)}\begin{cases}
%c_V=1\pm0.0098\\
%c_y=1\pm0.015\\
%c_g=0\pm0.016\\
%c_\gamma=0\pm0.047
%\end{cases}~~~~~~~
%\rho=\left(
%\begin{matrix}
%1 & 0.65 & -0.26 & 0.20\\
%0.65 & 1 & -0.41 & -0.32\\
%-0.26 & -0.41 & 1 & 0.13\\
%0.20 & -0.32 & 0.13 & 1
%\end{matrix}
%\right)
%\ea

To better quantify the sensitivity of the future LHC upgrades on CH models we identify the benchmark scenarios shown in table~\ref{benchark}. SILH1 (a,b) is intended to capture the low energy physics of CH models with a Nambu-Goldstone boson $H$, where couplings to the massless gauge bosons are suppressed~\cite{Giudice:2007fh}. SILH2 (a,b) is expected to mimic scenarios with an accidentally light CH doublet, since in that case one typically expects $|\bar c_{g,\gamma}|$ of order unity. To assess the impact of the fermionic coupling $\bar c_y$ we distinguished between scenarios with $\bar c_y=0$ (a) and $\bar c_y=1$ (b); models in which the SM fermions have different couplings for the various $\psi=u,d,e$ should lie somewhat in between these two. In table~\ref{benchark} we present, for each benchmark model, the expected HL/HE sensitivity. We treat the systematic uncertainties using the conservative hypothesis S1. Note the significant impact of a non-vanishing $\bar c_{g,\gamma}$ --- especially the coupling to gluons; of the four options with $\bar c_{g,\gamma}=\pm1$ we quote only the most stringent bound for brevity. 




We do not report the constraints from current LHC data because making a fair comparison with our projections is not straightforward. Future sensitivities on $c_{V,y}$ are expected to be a factor of 3-5 better than the present ones; however, current data slightly favors values $\bar c_{V,y}>1$, a fact that in some benchmark models results in stronger constraints than those shown in the HL-LHC column of table~\ref{benchark} (even when restricting our fit to the domain $0<\xi<1$).~\footnote{Perhaps a better measure of the improvement expected from the HL/HE-LHC can be obtained if we artificially assume the currently preferred central values are $c_V=c_y=1, c_g=c_\gamma=0$, as in our projections. In this latter case we find that at the $95\%$ CL the current LHC data lead to $\xi<0.095, 0.064, 0.019, 0.018$ for SILH1a,1b,2a,2b respectively.}





%%%%%%%%%%%%%%%%%%%%%
\begin{table}[t]
\caption{\small Projected HL/HE sensitivity (95\% CL) on the SILH parameter $\xi=v^2/f^2$ (and on $f$ in TeV) in our benchmark scenarios. Systematic uncertainties are treated according to the conservative scenario S1.
\label{benchark}}
\begin{center}
%\resizebox{\textwidth}{!}
{
\begin{tabular}{c|cccc||cc|cc} 
\rule{0pt}{1.2em}%

& $\bar c_H$ & $\bar c_y$  & $|\bar c_{g}|$ & $|\bar c_{\gamma}|$ & $\xi$ HL & $f$ HL &  $\xi$ HE & $f$ HE \\
\hline
\hline
SILH1a & $1$ & $0$ & $0$ & $0$ & $0.061$ & $1.0$ & $0.036$ & $1.3$ \\
SILH1b &$1$ & $1$ & $0$ & $0$ & $0.025$ & $1.6$ & $0.016$ & $2.0$ \\
\hline
SILH2a &$1$ & $0$ & $1$ & $1$ & $0.018$ & $1.8$ & $0.012$ & $2.2$\\
SILH2b & $1$ & $1$ & $1$ & $1$ & $0.014$ & $2.1$ & $0.0088$ & $2.6$ \\
\end{tabular}
}
\end{center}
\end{table}
%%%%%%%%%%%%%%%%%%%%%%%





%%%%%%%%%%%%%%%%%%%%%%
%\begin{table}[t]
%\begin{center}
%%\resizebox{\textwidth}{!}
%{
%\begin{tabular}{c|cccc|cc} 
%\rule{0pt}{1.2em}%
%
%& $\bar c_H$ & $\bar c_y$  & $|\bar c_{g}|$ & $|\bar c_{\gamma}|$ & $f$ [TeV] HL  &  $f$ [TeV] HE \\
%\hline
%\hline
%SILH1a & $1$ & $0$ & $0$ & $0$ & $1.0$ & $1.3$ \\
%SILH1b &$1$ & $1$ & $0$ & $0$ & $1.6$ & $2.0$ \\
%\hline
%SILH2a &$1$ & $0$ & $1$ & $1$ & $1.8$ & $2.2$\\
%SILH2b & $1$ & $1$ & $1$ & $1$ & $2.1$ & $2.6$ \\
%\end{tabular}
%}
%\end{center}
%\caption{\small 95\% CL constraints on the SILH parameter $\xi$ in our benchmark scenarios for HL-LHC and HE-LHC. Systematic uncertainties are treated according to the conservative scenario S1.
%\label{benchark}}
%\end{table}
%%%%%%%%%%%%%%%%%%%%%%%%


%We begin with a discussion of the SILH scenario. 




Recalling our definition \eqref{xi} we can translate the results of table~\ref{SILH} into a lower bound on the new physics scale $m_*$ as a function of the size of the typical coupling $g_*$ of the exotic sector. The result is shown in Fig~\ref{SILH}. For presentation purposes, in the figure we only show the reach of the HL-LHC and HE-LHC on the benchmark models SILH1b (black) and SILH2b (red). The constrained region lies on the right of each line. The bounds tend to push the CH scenario towards the SM limit $\xi\to0$, obtained decoupling the new physics scale. We see that in the case the exotic dynamics is maximally strongly coupled ($g_*\sim4\pi$) the LHC will be able to indirectly access mass scales of order $20-30$ TeV.

%%%%%%%%%%%%%%%%%%
%%%%%%%%%%%%%%%%%%
\begin{figure}[t]
\begin{center}
\includegraphics[scale=0.7]{\main/section9/plots/SILH1.pdf}
\caption{The constraints of table~\ref{benchark} are interpreted as lower bounds on the new physics scale $m_*$ for a given coupling $g_*$ of the strong dynamics, see \eqref{xi}. The blue lines define lower bounds on $m_*$ from current EW precision tests for two different assumptions on the UV dynamics (see text). The grey region identifies the unphysical regime $\xi>1$. 
}\label{SILH}
\end{center}
\end{figure}
%%%%%%%%%%%%%%%%%%
%%%%%%%%%%%%%%%%%%



In Fig~\ref{SILH} we also include (blue dashed and dot-dashed lines) the current $95\%$ CL limits derived from precision EW data~\cite{Baak:2014ora} and encoded in the oblique parameters ($U=0$)
\begin{eqnarray}
&&{\widehat S}=(1-c_V^2)\frac{g^2}{96\pi^2}\ln\frac{m_*}{m_Z}+{\widehat S}_{\rm UV}%\to-(c_V-1)\frac{1}{3\pi}\ln\frac{m_*}{m_Z}
\\\nonumber
&&{\widehat T}=-(1-c_V^2)\frac{3g'^2}{32\pi^2}\ln\frac{m_*}{m_Z}+{\widehat T}_{\rm UV}.%\to+(c_V-1)\frac{3}{4\pi c_w^2}\ln\frac{m_*}{m_Z}
\end{eqnarray}
Note that these include 1-loop effects within \eqref{eq:kappa.EFT.2} as well as contributions from heavy particles of mass $\sim m_*$ that we parametrized via $\widehat S_{\rm UV}={m_W^2}/{m_*^2}$ and $\widehat T_{\rm UV}$. The blue dot-dashed line refers to scenarios in which additional violations of custodial symmetry are negligible, $\widehat T_{\rm UV}=0$, whereas the blue dashed line to the more natural expectation $\widehat T_{\rm UV}=\xi 3y_t^2/(16\pi^2)$. Precision EW data already exclude a sizable portion of parameter space. However, as our plot clearly illustrates, these {\emph{indirect}} bounds significantly depend on unknown physics at the cutoff scale $m_*$. Hence, a {\emph{direct}} probe of the Higgs couplings always provides a more robust and model-independent assessment of the viability of a given CH scenario.








\subsubsection{The SILD}

The dominant interactions of the dilaton $h$ to the SM are derived from an EFT with non-linearly realized EW symmetry, where the NGBs eaten by the $W^\pm,Z^0$ are encapsulated into the unitary matrix $\Sigma$, which transforms as $\Sigma\to U_w\Sigma U_Y^\dagger$ under $SU(2)_w\times U(1)_Y$. The powers of the singlet $h$ are fully determined by the approximate conformal symmetry. %~\footnote{Each SM field is associated to a certain {{conformal weight}}. In particular $\Sigma$ has weight $0$. Note also that the derivative $\partial_\mu$ acting on an operator increases its weight by one unit. The dilaton field appears in a combination $(1+h/f_D)$ of weight 1 and is introduced multiplying each operator of weight $d$ in the EFT by a factor of $(1+h/f_D)^{4-d}$ in order to obtain a Lagrangian density of weight 4 (and the corresponding action a singlet under the conformal group).} 
Following the rules of non-linearly realized conformal symmetry, and neglecting possible (small) sources of custodial symmetry breaking, one identifies the dominant interactions:~\cite{Goldberger:2008zz}
\begin{eqnarray}\label{SILD}
{\cal L}_{\rm SILD}&=&\frac{v^2}{4}{\rm tr}[D_\mu\Sigma^\dagger D^\mu\Sigma]\left(1+\frac{h}{f_D}\right)^2-\sum_{\psi=u,d,l}m_\psi\bar\psi_L\Sigma\psi_R\left(1+\frac{h}{f_D}\right)^{1+\gamma_\psi}\\\nonumber
&+&\frac{g_s^2}{16\pi^2}\delta_sG^a_{\mu\nu}G^{a\mu\nu}\frac{h}{f_D}+\frac{e^2}{16\pi^2}\delta_e\gamma_{\mu\nu}\gamma^{\mu\nu}\frac{h}{f_D}\\\nonumber
&+&\cdots,
\end{eqnarray}
where the dots refer to operators that impact negligibly our analysis. In the unitary gauge, the first operator in \eqref{SILD} describes the main coupling between $h$ and the EW vector bosons. The couplings to fermions depend on how the latter interact with the underlying (approximately) conformal dynamics and are therefore model-dependent. A family-universal interaction in \eqref{SILD} is expected in any UV theory that does not feature sizable flavor-changing effects, which would otherwise be in tension with precision flavor observables. %In practice we will only consider scenarios with a flavor-independent $\gamma_{u,d,e}=\gamma_y$, analogously to what we have assumed for the SILH scenario. 
Interactions with the unbroken gauge bosons in the second line of \eqref{SILD} also depend on the details of the UV dynamics. We do not make any restrictive assumption here (besides an approximate CP symmetry) and instead allow the parameters $\delta_{g,\gamma}$ to acquire any real value. We however included a loop factor to emphasize we expect them to arise at the loop level. Similarly to what we have already stressed above \eqref{SILH}, novel corrections to $\gamma Z$ are not important to our analysis and can be ignored in a first assessment: the Lagrangian \eqref{SILD} is enough to capture the dominant on-shell signatures of the SILD scenario as well. From \eqref{SILD} one obtains a Lagrangian like \eqref{eq:kappa.EFT.2} with 
\begin{eqnarray}\label{4parSILD}
c_V=\frac{v}{f_D},~~~~~c_y=(1+\gamma_y)\frac{v}{f_D},~~~~~c_g=2\delta_g\frac{v}{f_D},~~~~~c_\gamma=\delta_\gamma\frac{v}{f_D}.
\end{eqnarray}
Under the simplifying assumption of flavor universality ($\gamma_{u,d,e}=\gamma_y$) our EFT contains only 4 independent real parameters. %Consistently with our assumption that $f_D$ denotes the scale of spontaneous (approximate) conformal symmetry of the strong Higgs sector, we will restrict our attention to the regime $f_D\geq v$. 


We are now able to draw a few qualitative conclusions. First, as anticipated experimental constraints on $c_V$ force $f_D\simeq v$, see Fig. \ref{SILD}. Therefore the SILD, as any other framework based on the non-linear chiral Lagrangian (i.e. where $h$ is not part of a Higgs doublet), implies the characteristic mass of the new physics lies at the relatively low scale $g_*v\lesssim4\pi v\sim3$ TeV. On the one hand this is an exciting possibility because it suggests its UV completion is more likely to be accessible at the LHC. On the other hand, a strong dynamics at such low scales contributing to EW symmetry breaking tends to be in serious trouble with EW precision data (cf. to $\widehat S_{\rm UV}\sim g^2/(16\pi^2)$ as in technicolor). One can only hope the UV theory somehow cures this problem, though no concrete mechanism to achieve this is known.


%%%%%%%%%%%%%%%%%%
%%%%%%%%%%%%%%%%%%
\begin{figure}[t]
\begin{center}
\includegraphics[scale=0.6]{\main/section9/plots/SILD.pdf}
\caption{Expected sensitivity of the HL-LHC (thick lines) and HE-LHC (thin lines) on SILD scenarios with $\delta_\gamma=0$ (black) or $\delta_g=0$ (red). For simplicity we assumed $\gamma_y=0$ in both cases.
}\label{SILD}
\end{center}
\end{figure}
%%%%%%%%%%%%%%%%%%
%%%%%%%%%%%%%%%%%%


Secondly, explicit realizations of the SILD are characterized by sizable couplings to the massless gauge bosons, $|\delta_{g,\gamma}|={\cal O}(1)$, and this appears to be in conflict with data. For example, the prototypical SILD scenario --- in which the entire SM is part of the conformal sector --- has $\delta_{g}=+46/3, \delta_\gamma=-34/9$~\cite{Goldberger:2008zz} and is already excluded with large confidence. Scenarios with partially composite SM, $|\delta_{g,\gamma}|\sim0.1$, will be subject to the significant constraints from the HL and HE upgrades, see Fig. \ref{SILD}. Unfortunately, even in this case there is no known symmetry argument that ensures $|\delta_{g,\gamma}|\ll1$. The only way to enforce it within explicit strongly-coupled models seems to be via fine tuning.



%Overall, the SILD scenario suffers from a number of drawbacks compared to the SILH. The key problem is that decoupling of the new physics scale is not an option here. The SM model is recovered in the limit $f_D\to v$ and $\gamma_{y},\delta_{g,\gamma}\to0$, which is equivalent to requiring a number of non-trivial conditions to be met simultaneously: the EW scale should parametrize the main source of conformal symmetry breaking ($f_D\simeq v$), the SM fermions should be weakly-coupled to the exotic sector ($|\gamma_y|\ll1$), and the new states should not modify the Higgs couplings to the gauge bosons ($|\delta_{g,\gamma}|\ll1$). Finding strongly-coupled incarnations of the SILD scenario with these properties apparently requires the tuning of several parameters (as opposed to the SILH, where $\xi\to0$ was sufficient to approach the SM limit). %In this sense, any realization of the SILD scenario that is compatible with data appears to be non-generic.

Overall, the SILD scenario suffers from a major drawback compared to the SILH: it does not possess a mechanism (analogous to $\xi\to0$) to decouple the new physics corrections in EW precision data as well as $c_{V,y,g,\gamma}$. If no new physics at the TeV scale is observed and no deviation from the SM is seen, directly or indirectly, in the future LHC upgrades, then we would have gathered enough evidence to conclude that the Higgs boson (irrespective of whether it is composite or not) must be part of an EW doublet. 




