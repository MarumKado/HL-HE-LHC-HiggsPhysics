\paragraph{Model-dependent limits}
\label{sec:model_dep}
%
At the tree level, the Higgs sector of the MSSM can be specified by suitable choices for two variables, 
often chosen to be the mass $m_{\PA}$ of the pseudoscalar Higgs boson
and $\tan\beta$, the ratio of the \mbox{vacuum} expectation
values of the two Higgs doublets.
The typically large radiative corrections are 
fixed based on experimentally and phenomenologically sensible choices for
the supersymmetric parameters, each choice defining a particular benchmark scenario.
Generally, MSSM scenarios assume that the 125 GeV Higgs boson is the lighter scalar \Ph, 
an assumption that is compatible with the current experimental constraints 
for at least a significant portion of the $m_{\PA}$--$\tan\beta$ parameter space.
The di-tau lepton final state provides the most sensitive direct search for additional
Higgs bosons predicted by the MSSM for intermediate and high values of tan$\beta$, 
because of the enhanced
coupling to down-type fermions.
%
\begin{figure}[htbp]
\begin{center}
\subfloat[$\mhmodp$]{\includegraphics[width=0.47\textwidth]{\main/section9/plots/mssm_mhmod_jul31_scen2_pas.pdf}}\\
\subfloat[hMSSM]{\includegraphics[width=0.47\textwidth]{\main/section9/plots/mssm_hmssm_jul31_scen2_pas.pdf}}
\subfloat[tau-phobic]{\includegraphics[width=0.47\textwidth]{\main/section9/plots/mssm_tauphobic_aug03_scen2_pas.pdf}}
\end{center}
\caption{Projection of expected MSSM \htt 95\% CL upper limits based on 2016 data~\cite{HIG-17-020} for different benchmark 
scenarios, with YR18 systematic uncertainties~\cite{CMS-PAS-FTR-18-017}. The limit shown for 6000\fbinv is an approximation of the sensitivity with 
the complete HL-LHC dataset to be collected by the ATLAS and CMS experiments, corresponding to an integrated luminosity of 3000\fbinv each. 
The limits are compared to the CMS result using 2016 data~\cite{HIG-17-020}; for the tau-phobic scenario, 
it is a new interpretation of the information given in this reference. 
}
\label{fig:model_mssm1}
\end{figure}

The analysis results are interpreted in terms of these benchmark scenarios based on the profile likelihood ratio of the 
background-only and the tested signal-plus-background hypotheses. 
For this purpose, the predictions from both production modes and both heavy neutral Higgs bosons are combined.
Figure~\ref{fig:model_mssm1} shows the results~\cite{HIG-17-020} 
for three different benchmark scenarios:
the $\mhmodp$ and tau-phobic scenarios~\cite{Carena:2013ytb} and the hMSSM~\cite{Djouadi:2013uqa,Bagnaschi:2015hka}.
The sensitivity reaches up to Higgs boson masses
of 2 TeV for values of $\tan \beta$ of 36, 26, and 28
for the $\mhmodp$, the hMSSM, and the tau-phobic scenarios,
respectively.
Even at low mass, improvements are expected but in this case they are mostly 
a consequence of reduced systematic uncertainties and not 
the additional data in the signal region.