\paragraph{Projection methodology}
\label{sec:method}
%
Three scenarios are considered for the projection of the size of
systematic uncertainties to the HL-LHC:
\begin{itemize}
\item
statistical uncertainties only: all systematic uncertainties are neglected;
\item
  Run 2 systematic uncertainties: all systematic uncertainties are held
  constant with respect to luminosity, i.e., they are assumed to be
  the same as for the 2016 analysis;
\item
  YR18 systematic uncertainties: systematic uncertainties are assumed to
  decrease with integrated luminosity
  following a set of assumptions described below.
\end{itemize}

In the YR18 scenario,
selected systematic uncertainties decrease
as a function of luminosity until they reach a certain minimum value. 
Specifically, all pre-fit uncertainties of an experimental nature
(including statistical uncertainties in control regions and
in simulated event samples) 
are scaled proportionally to the square root of the integrated luminosity.
The following minimum values are assumed:
\vspace{-0.2cm}
\begin{itemize}
\item
muon efficiency: 25\% of the 2016 value, corresponding to an average absolute uncertainty of about 0.5\%; 
\item
electron, $\tauhad$, and b-tagging efficiencies: 50\% of the 2016 values, 
corresponding to average absolute uncertainties
of about 0.5\%, 2.5\%, and 1.0\%, respectively;
\item
jet energy scale: 1\% precision for jets with $\pt > 30$ GeV; %, driven primarily by improvements for the absolute scale and jet flavour calibrations
\item
  estimate of the background due to jets misreconstructed
  as $\tauhad$~\cite{HIG-15-007},
  for the components that are not statistical in nature:
  50\% of the 2016 values;
%  determined using a fake factor method~\cite{HIG-15-007}, 
\item
luminosity uncertainty: 1\%;
\item
  theory uncertainties: 50\% of the 2016 values,
  independent of the luminosity for all projections.
\end{itemize}
Note that for limits in which the Higgs boson mass is larger than about 1 TeV,
the statistical uncertainties dominate and
the difference between the systematic uncertainties found from
the different methods has a negligible impact on the results.

The lightest Higgs boson, \Ph, is excluded from the SM versus MSSM hypothesis test for the following reason:
With increasing luminosity, the search will become sensitive to this boson. However, the current benchmark scenarios do not
incorporate the properties of the \Ph boson with the accuracy required at the time of the HL-LHC.
Certainly the benchmark scenarios will evolve with time in this respect. Therefore the signal hypothesis includes
only the heavy $\PA$ and \PH bosons, to demonstrate the search potential only for these.