\subsubsection{Extrapolation assumptions}
\label{sec:HiggsExtrapAss}

%% \textbf{Do we need this introductory text, or will it be discussed in Section 1? In case, it needs to be updated with the corresponding CMS information. --M.D. 2018-11-07} \\

The results presented in this Section are based on the extrapolation to an expected integrated luminosity of 3000~fb$^{-1}$ at $\sqrt{s}$ = 14 TeV of the corresponding ATLAS and CMS Run-2 results. For some of the Higgs decay final states (ATLAS: $WW^*$, $Z\gamma$, $t\bar{t}H$, $\tau\tau$; CMS: $b\bar{b}$) the extrapolation is performed on results obtained with the 2015-2016 36\,$\ifb$ datasets; the remaining final state analyses (ATLAS: $\gamma\gamma$, $ZZ^*$, $b\bar{b}$ and $\mu\mu$) use the results based on the 2015+2016+2017 80\,$\ifb$ data samples. The starting points of the extrapolated results are measurements based on datasets of size $\mathcal{O}(1\%)$ of the expected HL-LHC integrated luminosity. The extrapolations are in this respect very limited with respect to the potential reach of the real HL-LHC analyses, which large statistics will allow to probe corners of the phase space inaccessible at the LHC Run-2.
    
In addition to the increase in integrated luminosity, in most of the studies the extrapolations also account for the increase of signal and background cross-sections from $\sqrt{s}$ = 13 TeV to 14 TeV.  In those cases, the signal yields have been scaled according to the Higgs boson production cross sections values at 13 and 14 TeV, as reported in Ref.~\cite{deFlorian:2016spz}. Similarly, the background yields have been scaled according to the parton luminosity ratio between 13 and 14 TeV, as reported in Ref.~\cite{Heinemeyer:2013tqa}, by taking into account whether the background process is predominantly quark pair or gluon pair initiated.

Object reconstruction efficiencies, resolutions and fake rates are assumed to be similar in the Run-2 and HL-LHC environments, based on the assumption that the planned upgrades of the ATLAS and CMS detectors will compensate for the effects of the increase of instantaneous luminosity and higher pile-up environment at HL-LHC.
%% Experimental uncertainties related to object reconstructions are reduced, neglected or kept untouched depending on the their sources.
For the systematic uncertainties which include experimental, signal and background components, two scenarios have been considered.
The first scenario (S1) assumes the same values as those used in the published Run-2 analyses.
The second scenario (S2) implements a reduction of the systematic uncertainties according to the improvements expected to be reached at the end of HL-LHC program in twenty years from now: the correction factors follow the recommendations from Ref.~\cite{HLHELHCCommonSystematics}.
In certain analyses some of the systematic uncertainties are treated in a specific way, and this is discussed explicitly in each corresponding section.
%
In all analyses, the theory uncertainties for signal and background are generally halved, except where more precise extrapolated values have been provided. Details on the projections of theoretical uncertainties are given in Section~\ref{sec:hl-lhc}. The reduction of the theory uncertainties in gluon-fusion Higgs production is for instance associated to a better understanding of their correlation of their components, leading to their sum in quadrature in scenario S2, instead of the linear sum used in S1 (see Section~\ref{sec:hl-lhc-ggF} for details). The uncertainties related to the PDF are in particular discussed in Section~\ref{sec:2:PDFuncertainties}: these uncertainties are halved in all analyses extrapolation in scenario S2, even though some larger improvements are expected in some cases (e.g. gluon-fusion Higgs production).
%
The uncertainty on the luminosity is set to 1\%.
The uncertainty related to Monte Carlo samples statistics is assumed to be negligible.
%% In the relevant analyses, the spurious signal uncertainty, accounting for backgroundfunctional mismodelling, is also assumed to become negligible.

The extrapolated results are generally limited by systematic uncertainties. It is worth noting that, despite all efforts to design proper projections, the values of the systematic uncertainties of the Run-2 analyses cannot fully account for the HL-LHC conditions and process understanding. The systematic models in current Run-2 analyses are in fact designed for the needs of Run-2, and hence lack flexibility and details needed to account for full-fledged HL-LHC analyses. In this sense, these extrapolated uncertainties are to be considered an approximation. Future analyses will exploit and gain sensitivity from phase space regions that are not accessible yet, or use analysis techniques that reduce the impact of systematic uncertainties.

In the following, all analyses segment the selected events according to the objects produced in association with the Higgs boson decay products and their topology, in order to maximize the sensitivity to the main Higgs production modes ($ggH+b\bar{b}H$, VBF, $VH$ = $qqZH+ggZH+WH$ and top = $t\bar{t}H+tH$)  and to reduce the uncertainties on the respective cross sections. Details on how this segmentation is performed, and on the event selection and categorisation in the various analyses, are found in the Run-2 analysis references quoted in each section.