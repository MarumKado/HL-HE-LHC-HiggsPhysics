%\begin{center}
%\textit{by S. Boselli, C. M. Carloni Calame, G. Montagna,
%  O. Nicrosini, F. Piccinini, A. Shivaji, F. Yu, Maria Moreno Llacer
%  et al.}
%\end{center}
%
 We present prospects for studies on {\it CP}-odd couplings in the
 interactions of the Higgs boson with the electroweak gauge bosons as well
 as in the Yukawa couplings of the Higgs boson with fermions, in
 particular with $\tau^+ \tau^-$ pairs.
 
 \subsubsubsection{\bf CP-odd $VVH$ couplings}

While a large number of studies assessing the impact of {\it CP}-even  
effective operators on Higgs physics is available in the literature 
(see for instance our analysis in Ref.~\cite{Boselli:2017pef} and the references therein), 
the present analysis is focused on the impact of {\it CP}-odd effective operators on the 
interactions among the Higgs boson and the electroweak bosons. 
In the Higgs basis, the CP-violating (CPV) sector of the BSM Lagrangian affecting $VVH$ couplings 
is given by, 
% 
\begin{eqnarray}
\lag_{\rm CPV} =  \frac{H}{v} \Big[
\tCaa \frac{e^2}{4} A_{\mu\nu}\tilde{A}^{\mu\nu}  
+ \tCza \frac{e\sqrt{g_1^2 + g_2^2}}{2} Z_{\mu\nu}\tilde{A}^{\mu\nu} 
+ \tCzz \frac{g_1^2 + g_2^2}{4} Z_{\mu\nu}\tilde{Z}^{\mu\nu} + \tilde{c}_{WW} \frac{g_2^2}{2}  W^+_{\mu\nu} \tilde W^{-\mu\nu}\Big]
% \left. \vphantom{\half} \rq.
\label{eqn:Hvv}
\end{eqnarray}  
where, $g_1$ and $g_2$ are the $U(1)_Y$  and  $SU(2)_L$ gauge coupling constants. Out of the above four 
parameters, only three  are independent. In particular,
\begin{equation}
 \tCww = \tCzz + 2s_\theta^2 \tCza + s_\theta^4 \tCaa.
\end{equation}

The processes which are sensitive to CP-odd operators are the Higgstrahlung processes ($WH$ and $ZH$), the vector boson fusion (VBF) and the Higgs decay into four charged leptons ($H\to4\ell$). Here we focus on angular observables which are sensitive to CPV effects. Indeed, since the total cross-section is a CP-even quantity,  the $1/\Lambda^2$ effects of CPV operators can affect the shape of some specific kinematic distributions only. 



\begin{table*}
        \begin{center}
                \begin{tabular}{lc|ccccc}
                        \multicolumn{2}{c|}{Process}     & Combination & Statistical & Theory (Sig.) & Theory (Bkg.) & Experimental \\ \hline \hline
                        \multirow{5}{*}{${H}\to {ZZ}$}
                        &$\text{ggF}$      & 6.6  & 2.1  & 5.4  & 1.7 & 2.7\\
                        &$\text{VBF}$      & 15.2 & 11.7 & 9.1  & 2.4 & 1.8\\
                        &${WH}$            & 48.0 & 46.5 & 6.2  & 2.8 & 7.8\\
                        &${ZH}$            & 82.5 & 75.7 & 27.0 & 7.6 & 16.4\\
                        &${t\overline tH}$ & 26.9 & 23.6 & 10.9 & 2.5 & 4.2\\ \hline
                \end{tabular}
                \caption{Estimated uncertainties [\%] on the determination of single-Higgs production channels in $H \to 4\ell$ decay mode. These are CMS projections for high-luminosity LHC (14 \UTeV centre of mass energy and 3 ab$^{-1}$ integrated luminosity) in scenario S1 (systematic uncertainties are kept constant with luminosity) taken from Ref.~\cite{CMS-PAS-FTR-18-011}.}\label{tab:one}
        \end{center}
\end{table*}

\subsubsubsection{\bf Global Fit}

To study the sensitivity on CP-violating parameters $\tCza$ and $\tCzz$ at HL and HE-LHC, we perform a $\chi^2$ fit using, as observable, 
the signal strength ($\mu_{i,f}$) in the Higgs production channel ($i$) and Higgs decay channel ($f$). 
We can build a $\chi^2$  as follows:

\begin{equation}
 \chi^2(\tCza,\tCzz) = \sum_{i,f} \frac{(\mu_{i,f} -\mu_{i,f}^{\rm obs.})^2}{\Delta_{i,f}^2}
\end{equation}
{
The signal strength, $\mu_{i,f}$ is a function of the BSM parameters and it is defined as, 
\begin{eqnarray}
 \mu_{i,f} &=& \mu_i \times \mu_f \\
         &=& \frac{\sigma_i^{\rm BSM}}{\sigma_i^{\rm SM}} \times \frac{{\rm BR}_f^{ \rm BSM}}{{\rm BR}_f^{ \rm SM}}.
\end{eqnarray}

The total uncertainty, $\Delta_{i,f}^2$ includes theoretical, experimental systematic and statistical uncertainties, which 
are added in quadrature.
The one-sigma uncertainties for the high-luminosity (14 \UTeV centre of mass energy and 3 ab$^{-1}$ integrated luminosity) are given in table~\ref{tab:one}. 
Assuming the same acceptance efficiency, we scale the statistical uncertainties at 14 \UTeV and 3 ab$^{-1}$ luminosity appropriately 
to obtain the statistical uncertainties at 27 \UTeV and 15 ab$^{-1}$ 
luminosity. The theoretical and experimental systematic uncertainties are kept unchanged.

When considering kinematic distributions in the fit, we estimate the statistical uncertainty in each bin by scaling 
the overall statistical uncertainty by the fraction of number 
of events in each bin. On the other hand, the theoretical 
and systematic uncertainties are assumed to be the same in all the bins implying 
a very conservative scenario.


Since we are interested in the sensitivity on the CPV parameters that can be reached at HL and HE LHC, due to the present lack of experimental data, we take 
$\mu_{i,f}^{\rm obs.}=1$, implying that the future data would be consistent with the SM hypothesis. In the current analysis, we consider all the single Higgs production channels and Higgs decaying to four charged-leptons, {\it i.e} $i={{\rm ggF, VBF}, ZH, WH, t{\bar t}H}$ and $f=4\ell (2e2\mu, 4e, 4\mu)$. The projected uncertainties in these channels for HL-LHC are given in table~\ref{tab:one}. 
%Note that only the $H\to 4\ell $ decay mode has a non-trivial kinematic distribution and therefore other decay modes in the present analysis have been ignored.
All the results in the following sections are presented taking $M_H=$125 \UGeV.
}\\

{\bf Production signal strengths : Inclusive} \\

{
The first step is to calculate the signal strengths for the relevant production channels in presence of the CP-violating
parameters $\tCza$ and $\tCzz$. We use {\tt Madgraph5\_aMC@NLO}~\cite{Alwall:2014hca} to obtain the inclusive cross sections in presence of these parameters. We have generated the required UFO model file for {\tt Madgraph} using the {\tt FeynRules} package~\cite{Degrande:2011ua,Alloul:2013bka}.  At 14 \UTeV, the production signal strengths are given by, }


\begin{eqnarray}\label{eq:mu14tev}
 \mu_{ZH}^{\rm 14 \UTeV} &=& 1.00 +  0.54~ \tCza^2 + 2.80~ \tCzz^2 + 0.95~ \tCza \tCzz \\
 \mu_{WH}^{\rm 14 \UTeV} &=& 1.00  + 0.84~ \tCza^2 + 3.87~ \tCzz^2 
   + 3.63~\tCza\tCzz \\
 \mu_{\rm VBF}^{\rm 14 \UTeV} &=& 1.00  + 0.25~ \tCza^2 + 0.45~ \tCzz^2  
   + 0.45~\tCza\tCzz
\end{eqnarray}




{ At 27 \UTeV, the corresponding signal strengths are given by,}

\begin{eqnarray}\label{eq:mu27tev}
 \mu_{ZH}^{\rm 27 \UTeV} &=& 1.00 +  0.63~ \tCza^2 + 3.26~ \tCzz^2 + 1.11~ \tCza \tCzz \\
 \mu_{WH}^{\rm 27 \UTeV} &=& 1.00 + 0.98~ \tCza^2 + 4.48~ \tCzz^2 
  + 4.16~\tCza\tCzz \\
 \mu_{\rm VBF}^{\rm 27 \UTeV} &=& 1.00  + 0.32~ \tCza^2 + 0.67~ \tCzz^2  
  + 0.65~\tCza\tCzz
\end{eqnarray}

The BSM predictions for VBF are derived using following cuts, 
\begin{equation}
 p_T(j) > 20~{\rm \UGeV}, |\eta(j)| < 5, \Delta\eta_{jj} > 3, m_{jj} > 130~{\rm \UGeV} \nonumber.
\end{equation}
We find that the $VH$ production modes are more sensitive to $\tCzz$ parameters.
The ggF and $t{\bar t}H$ production channels are unaffected in presence of CP-violating $VVH$ 
couplings. Therefore, 
% 
\begin{eqnarray}
 \mu_{\rm ggF}^{\rm 14 \UTeV} &=& \mu_{\rm ggF}^{\rm 27 \UTeV} =  1.00 \\
 \mu_{t \bar t H}^{\rm 14 \UTeV} &=& \mu_{t\bar t H}^{\rm 27 \UTeV} =  1.00 .
\end{eqnarray}

{ In the present analysis we consider only kinematic distributions of the Higgs decay products, 
in the Higgs rest frame.} \\

{\bf Decay signal strength : Inclusive} \\


Now we turn to the calculation of the signal strength for the decay channel $H \to 4\ell$. This decay 
channel receives contributions from $2e^+2e^-$ ($4e$), $2\mu^+ 2\mu^-$ ($4\mu$) and $e^+e^-\mu^+\mu^-$ ($2e2\mu$) final states.
We use the latest version of the {\tt Hto4l} event generator~\cite{Boselli:2017pef} to obtain the partial decay widths in these 
channels in presence of $\tCza$ and $\tCzz$. Both the $e$ and $\mu$ are treated as massless. The ratio of the partial decay widths in BSM and in SM ($R_\Gamma$) for different channels are given by, 

{
\begin{eqnarray}
 R_{\Gamma}(H \to 2e2\mu) &=& 1  + 1.174~ \tCza^2 + 0.00291~ \tCzz^2  
   + (-0.00762)~\tCza\tCzz \\
 R_{\Gamma}(H \to 4e) 
 &=& R_{\Gamma}(H \to 4\mu) \nonumber \\
 &=& 1  + 1.106~ \tCza^2 + 0.00241~ \tCzz^2 
   + (-0.00595)~\tCza\tCzz.
\end{eqnarray}
}
The above expression for Higgs decay
into $2e2\mu$ and $4e$ are obtained after applying a selection cut of { 4 \UGeV} on the leading and sub-leading lepton pairs
of opposite sign.

In the present analysis, we also assume that the total Higgs decay width remains unchanged in presence of BSM. In this
case, the signal strength for decay is just the ratio of decay widths in BSM and in SM, that is,
\begin{eqnarray}\label{eq:mu4l}
 \mu_{4\ell} &=& \frac{\Gamma^{\rm BSM}_{4\ell}}{\Gamma^{\rm SM}_{4\ell}}  \nn \\
 &=& 1  + 1.138~ \tCza^2 + 
 0.00265~ \tCzz^2  +
 (-0.00674)~ \tCza \tCzz 
\end{eqnarray}
We note that, the dependence of the $4\ell$ decay signal strength on the parameter $\tCzz$ is very weak. \\

{\bf Decay signal strength : Differential} \\

{
We now turn to assessing  the role of kinematic distributions in $H \to 4\ell$ decay channel, 
which are affected by CP-violating $VVH$ couplings, in improving the sensitivity on $\tCza$ and $\tCzz$ at the HL and HE-LHC.
% The kinematic variable considered in the present analysis is,
% \begin{equation}
%  	\cos \phi =
% 	\frac{\left( k_{12} \times k_1\right) \cdot \left( k_{12} \times k_3 \right)}
% 	{\left| k_{12} \times k1 \right| \left| k_{12} × k_3 \right|}
% \end{equation}
The Higgs rest frame angle $\phi$ between the decay planes of the two intermediate gauge bosons  
is very sensitive to the CP-Violating $VVH$ couplings~\cite{Soni:1993jc,Chang:1993jy,Skjold:1993jd,Buszello:2002uu}. We have considered 50 bins of $\phi$-distribution to perform the fit at differential level. For each bin, we calculate the signal strength ($\mu_{4\ell,j}; j=1\to50$) corresponding 
to Eq.~\ref{eq:mu4l}. Unlike $\mu_{4\ell}$ in Eq.~\ref{eq:mu4l}, $\mu_{4\ell,j}$ is also sensitive to linear terms in $\tCza$ and $\tCzz$.

\subsubsubsection{\bf Result: HL and HE-LHC Analyses}

\begin{figure}[h!]
\centering
 \includegraphics[scale=0.6]{\main/section2/plots/tCza-Chisq-HL3ab-HE15ab-S1}
\includegraphics[scale=0.6]{\main/section2/plots/tCzz-Chisq-HL3ab-HE15ab-S1}
\caption{ $\chi^2$ dependence on CP-violating parameters taking one parameter non-zero at a time 
at HL-LHC (3 ab$^{-1}$, green) and HE-LHC (15 ab$^{-1}$, blue) for uncertainty scenario S1. The solid lines refer to the fit performed using $H \to 4\ell$ decay width at inclusive level (1 bin) while, the dashed lines refer to the fit obtained using $H \to 4\ell$ decay width at differential level ($\phi$-distribution with 50 bins).}\label{fig:fit1p}
\end{figure}

\begin{figure}
\centering
 \includegraphics[scale=1.2]{\main/section2/plots/tCza-tCzz-HL3ab-HE15ab-S1}
\caption{ $1\sigma$  reach on $\tCza$ and 
$\tCzz$ at HL-LHC (3 ab$^{-1}$, green) and HE-LHC (15 ab$^{-1}$, blue) for uncertainty scenario S1. The solid lines refer to the fit performed using $H \to 4\ell$ decay width at inclusive level (1 bin) while, the dashed lines refer to the fit obtained using $H \to 4\ell$ decay width at differential level ($\phi$-distribution with 50 bins).  }\label{fig:fit2p}
\end{figure}

The results of the $\chi^2$ fit for CP-violating parameters $\tCza$ and $\tCzz$ are displayed in Fig.~\ref{fig:fit1p} and Fig.~\ref{fig:fit2p}. In these results, 
 {\it incl.} refers to the fit obtained using 
the partial decay width information in the $H \to 4\ell$ channel, while {\it diff.} refers to the fit obtained using 
$\phi$-distribution in $H \to 4\ell$ decay. In Fig.~\ref{fig:fit1p}, we show 
$1\sigma$ and $2\sigma$ bounds on $\tCza$ and $\tCzz$ in a one parameter (1P) 
analysis.  We find that at HL-LHC we are more sensitive to $\tCza$ than to $\tCzz$. At the inclusive level 
we gain better sensitivity on $\tCzz$ than on $\tCza$ when going from HL-LHC 
to HE-LHC. This is mainly due to a stronger dependence of the production signal strength on parameter $\tCzz$. However, due to a stronger dependence of $\mu_{4\ell}$ on $\tCza$ the effect of using $\phi$-distribution in the fit is larger for $\tCza$ than for $\tCzz$.

In Fig.~\ref{fig:fit2p}, we provide $1\sigma$ contour lines in the $\tCza-\tCzz$ plane. We can see that the parameters $\tCza$ and $\tCzz$ are weekly correlated. 
Once again we find that using $\phi$-distribution in the fit improves our 
sensitivity on CP-violating parameters significantly.
The parameter $\tCzz$ is mainly constrained by the production channels $VH$ and VBF.
We have given a summary of $1\sigma$ bounds on $\tCza$ and $\tCzz$ obtained from our analyses for HL and HE-LHC in Table~\ref{tab:tab3}.
}


\begin{table}
 \centering
 \begin{tabular}{l|cc|l}
 \hline
  \backslashbox{Analysis}{Parameter} & $\tCza$ & $\tCzz$ & Case \\
  \hline\hline
    HL-LHC ($4\ell$, incl.) & [-0.22,0.22] & [-0.33,0.33]& 1P \\
                            & [-0.25,0.25] & [-0.27,0.27]& 1P$_{marg.}$ \\
    \hline
    HL-LHC ($4\ell$, diff.) & [-0.10,0.10] & [-0.31,0.31]& 1P \\
                            & [-0.13,0.13] & [-0.22,0.22]& 1P$_{marg.}$ \\
    \hline
    HE-LHC ($4\ell$, incl.) & [-0.18,0.18] & [-0.17,0.17]& 1P \\
                            & [-0.23,0.23] & [-0.20,0.20]& 1P$_{marg.}$ \\
    \hline
    HE-LHC ($4\ell$, diff.) & [-0.05,0.05] & [-0.13,0.13]& 1P  \\
                            & [-0.06,0.06] & [-0.10,0.10]& 1P$_{marg.}$ \\
 \end{tabular}
\caption{ Summary of 1$\sigma$ bounds on $\tCza$ and $\tCzz$ from various analyses considered in our study for uncertainty scenario S1. 1P refers to the case
where only one parameter is non-zero while 1P$_{marg.}$ refers to the case in which the effect of one of the two parameters is marginalised.}\label{tab:tab3}
\end{table}




\subsubsubsection{$h \to \tau^+ \tau^-$}

The most promising direct probe of CP violation in fermionic Higgs
decays is the $\tau^+ \tau^-$ decay channel, which benefits from a
relatively large $\tau$ Yukawa giving a SM branching fraction of
$6.3\%$. Measuring the CP violating phase in the tau Yukawa requires a measurement of the linear polarisations of both $\tau$ leptons and the azimuthal angle between them. This can be done by analysing tau substructure, namely the angular distribution of the various components of the tau decay products.

The main $\tau$ decay modes studied include $\tau^\pm \to
\rho^\pm (770) \nu$, $\rho^\pm \to \pi^\pm \pi^0$~\cite{Bower:2002zx,
  Desch:2003mw, Desch:2003rw, Harnik:2013aja, Askew:2015mda,
  Jozefowicz:2016kvz} and $\tau^\pm \to \pi^\pm
\nu$~\cite{Berge:2008wi, Berge:2008dr, Berge:2011ij}.  Assuming CPT
symmetry, collider observables for CP violation must be built from
differential distributions based on triple products of three-vectors.
In the first case, $h \to \pi^\pm \pi^0 \pi^\mp \pi^0 \nu \nu$,
angular distributions built only from the outgoing charged and neutral
pions are used to determine the CP properties of the initial $\tau$
Yukawa coupling.  In the second case, $h \to \pi^\pm \pi^\mp \nu \nu$,
there are not enough reconstructible independent momenta to construct an observable sensitive to CP violation, requiring additional kinematic information such as the $\tau$ decay impact parameter.

In the kinematic limit when each outgoing neutrino is taken to be
collinear with its corresponding reconstructed $\rho^\pm$ meson, the
acoplanarity angle, denoted $\Phi$, between the two decay planes
spanned by the $\rho^\pm \to \pi^\pm \pi^0$ decay products is exactly
analogous to the familiar acoplanarity angle from $h \to 4 \ell$
CP-property studies~\cite{Chatrchyan:2012jja, Aad:2013xqa}.  Hence, by measuring the $\tau$ decay products in
the single-prong final state, suppressing the irreducible $Z \to
\tau^+ \tau^-$ and reducible QCD backgrounds, and reconstructing the
acoplanarity angle of $\rho^+$ vs.~$\rho^-$, the differential
distribution in $\Phi$ gives a sinusoidal shape whose maxima and
minima correspond to the CP-phase in the $\tau$ Yukawa coupling.  We can parametrise the $\tau$ Yukawa coupling to include CP violation using the Lagrangian term $\frac{y_\tau}{\sqrt{2}} h \bar{\tau} (\cos \Delta + i \sin \Delta \gamma^5) \tau$, where $y_\tau$ is the magnitude of the tau Yukawa coupling and remains fixed to the SM value for this study.

An optimal observable using the collinear approximation was derived in~\cite{Harnik:2013aja}. Assuming 70\% efficiency for tagging hadronic $\tau$ final states, and
neglecting detector effects, the estimated sensitivity for the
CP-violating phase $\Delta$ of the $\tau$ Yukawa coupling using 3 ab$^{-1}$ at
the HL-LHC is 8.0$^\circ$.  A more sophisticated
analysis~\cite{Askew:2015mda} found that detector resolution effects
on the missing transverse energy distribution degrade the expected
sensitivity considerably, and as such, about 1 ab$^{-1}$ is required
to distinguish a pure scalar coupling (CP phase is zero) from a pure
pseudoscalar coupling (CP phase is 90$^\circ$).

At the HE-LHC, the increased signal cross section for Higgs production
is counterbalanced by the increased background rates, and so the main
expectation is that improvements in sensitivity will be driven by the
increased luminosity and more optimised experimental methodology.
Rescaling with the appropriate luminosity factors, the optimistic
sensitivity to the $\tau$ Yukawa phase from acoplanarity studies is
4-5$^\circ$, while the more conservative estimate is roughly an order
of magnitude bigger.

\subsubsubsection{$ t\,\bar{t}\,h$}

CP violation in the top quark-Higgs coupling is strongly constrained by EDM measurements and Higgs rate measurements~\cite{Brod:2013cka}. However, these constraints assume that the light quark Yukawa couplings and $hWW$ couplings are set to their SM values. If this is not the case, the constraints on the phase of the top Yukawa coupling are less stringent.
    
Assuming the EDM and Higgs rate  constraints can be avoided, the CP structure of the top quark Yukawa can be probed directly in $pp \to t\bar t h$. Many simple observables, such as $m_{t\bar t h}$ and $p_{T,h}$ are sensitive to the CP structure, but require reconstructing the top quarks and Higgs.

Some $t\bar t h$ observables have been proposed recently that access the CP structure without requiring full event reconstruction. These include the azimuthal angle between the two leptons in a fully leptonic $t/bar{t}$ decay with the additional requirement that the $p_{T,h} > 200\, \text{\UGeV}$~\cite{Buckley:2015vsa}, and the angle between the leptons (again in a fully leptonic $t/\bar t$ system) projected onto the plane perpendicular to the $h$ momentum~\cite{Boudjema:2015nda}. These observables only require that the Higgs is reconstructed and are inspired by the sensitivity of $\Delta \phi_{\ell^+\ell^-}$ to top/anti-top spin correlations in $pp \to t\bar t$~\cite{Mahlon:1995zn}. The sensitivity of both of these observables improves at higher Higgs boost (and therefore higher energy), making them promising targets for the HE-LHC, though no dedicated studies have been carried out to date.
