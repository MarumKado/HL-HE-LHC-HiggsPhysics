

The large number of events expected in almost all Higgs boson measurement channels for the HL-LHC and HE-LHC will allow very precise measurements of the Higgs boson production cross sections and its couplings to gauge bosons and fermions. In many measurement channels, the expected overall statistical, experimental and theoretical uncertainties will be comparable in size. Therefore, a close interaction between the communities of the experimental and theoretical particle physicists will be needed in order to reach the best possible measurements of the Higgs boson properties.

Experimental sensitivity for the Higgs boson properties measurements is estimated by extrapolating the performance of the existing measurements to the HL-LHC data set, assuming the experiments will have a similar level of detector and triggering performance. Results are presented for two assumptions on the size of the experimental and theoretical systematic uncertainties that will be achievable by the time of HL-LHC (so called conservative and optimistic scenarios). Details on the extrapolation methodology and scenarios will be presented in Section~\ref{sec:2:channels}.

Section~\ref{sec:2_HXSWG1} provides an overview of theoretical predictions for the Higgs boson production at 14 and 27~\UTeV\ and of the uncertainties that are expected to be reached by the time of the final HL-LHC and HE-LHC measurements. 
These predictions are used as input to sensitivity studies of the ATLAS and CMS Higgs boson cross section and coupling measurements in individual channels that are summarized in Section~\ref{sec:2:channels} and for the expectations for differential cross section measurements presented in Section~\ref{sec:2:fiducial}.
Section~\ref{sec:2:top} puts emphasis on all measurements related to the top Yukawa coupling, as this is the largest Yukawa coupling in the Standard Model with a value close to unity and, hence, of special interest in understanding the Higgs mechanism and its relation to fermions.
The combination of the expected measurements in ATLAS and CMS are presented in Section~\ref{sec:2:exp_combination} together with an interpretation in the kappa-model~\cite{LHCHiggsCrossSectionWorkingGroup:2012nn,Heinemeyer:2013tqa} in Section~\ref{sec:2:exp_kappa}.

The kappa-framework is closely related to a non-linear EFT as discussed in Section~\ref{sec:2:kappavsEFT} and projections of measurements of EFT coefficients in a non-linear EFT are presented in Section~\ref{sec:2:theo_kappa_EFT}. 
%Projections of Higgs boson measurements to the HE-LHC are presented in Section~\ref{sec:2:HE_LHC_Higgs_projections}. 
Finally, probes of anomalous HVV interactions are discussed in Section~\ref{sec:2:anomalous_HVV}.
