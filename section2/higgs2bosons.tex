\subsubsection{$H \to \gamma\gamma$}
\label{sec:Hgammagamma}
%% {\it To be written by: M. Delmastro}

The measurement of the Higgs boson properties in the \Hyy\ channel is extrapolated from the most recent measurements by ATLAS with 80\,$\mathrm{fb}^{-1}$ \cite{ATLAS:2018uso} and CMS with XX\,$\mathrm{fb}^{-1}$ \cite{}.
%%
Events are selected to contain two isolated photon candidates passing good quality requirements in the precision regions of the detectors.
%% Events are further segmented according to the objects campaigning the diphoton system, in order to maximize the sensitivity to the main Higgs production modes ($ggH+b\bar{b}H$, VBF, $VH$ = $qqZH+ggZH+WH$ and top = $t\bar{t}H+tH$) and to reduce the uncertainties on the respective cross sections, as well as to the Simplified Template Cross Section (STXS, first introduced in Refs. \cite{deFlorian:2016spz,Badger:2016bpw}) in the merged version of Stage-1.
The Higgs production cross sections are measured for a Higgs boson absolute rapidity $|y_H|$ smaller than 2.5, and with further requirements on the objects campaigning the diphoton system (e.g. jet $p_\mathrm{T}$).
%%
The \Hyy\ signal is extracted by means of a combined signal-plus-background fit of the diphoton invariant mass spectra in the various event categories, where both the continuous background and the signal resonance are parameterized by analytical functions. The shape properties of the signal PDF are obtained by MC simulation, and constrained by performance studies of the photon energy scale and resolution. The background PDF is completely determined by the fit on data, with systematic uncertainties attributed to the specific choice of functional form following the procedure described in Ref. \cite{Aad:2012tfa} or using the discrete profiling method \cite{Dauncey:2014xga}.

The main systematic uncertainties affecting the results are the background modelling uncertainty, QCD scale uncertainties causing event migrations between the bins, photon isolation efficiencies and jet uncertainties.
%% Beside that, the underlying event and parton showering uncertainties as well as the PES and PER play are role.
%
On top of the common assumptions mentioned in Section~\ref{sec:HiggsExtrapAss}, the ATLAS \Hyy\ results include a 10\% increase of the background modeling systematic uncertainties, to account for the potentially worst knowledge of the background composition in each analysis category at HL-LHC: this assumption has anyway negligible impact.
%
In the Run-2 analyses, a conservative 100\% uncertainty on the heavy flavour resonant background in top-sensitive categories is applied. Measurements by ATLAS and CMS of the heavy flavour content, or the $b$-jet multiplicity, are expected to better constrain these contributions: for the S2 scenario extrapolation, this uncertainty is therefore halved.

Figure~\ref{fig:Hyy_ATLAS_HLLHC_S2} show the ratio of the extrapolated \Hyy\ ATLAS measurements of the main four Higgs production modes to their respective theoretical SM predictions in the scenario S2. The reduction of the total uncertainty with respect to the 80\,$\mathrm{fb}^{-1}$ results ranges from a factor of about 2(3) for the S1 (S2) scenario for the $ggH+b\bar{b}H$, VBF, top cross sections, to a factor of about 5(6) for the $VH$ cross section.

\begin{figure}
  \centering
  \includegraphics[width=0.6\linewidth]{\main/section2/plots/channels/ATLAS_Hyy_compareToSM_prodXS}
  \caption{Ratio of the cross-section times branching fraction measurements of the main four Higgs production modes in the \Hyy\ decay channel to their respective theoretical SM predictions, as extrapolated at the HL-LHC for scenario S2 by ATLAS.}
  \label{fig:Hyy_ATLAS_HLLHC_S2}
\end{figure}

\subsubsection{$H \to Z\gamma \to 2\ell\,\gamma$}
%%{\it To be written by: M. Delmastro}

Due to the small branching fraction in the SM, the \HZy\ decay has not yet been observed at the LHC. The experimental observed limits at the $95\%$ confidence level are currently 6.6 times the SM prediction for a Higgs boson mass of 125.09 GeV by ATLAS, based on the analysis of 36.1\,$\mathrm{fb}^{-1}$ of $pp$ collision at $\sqrt{s} = 13$ TeV \cite{HIGG-2016-14}, and XX times the SM prediction by CMS, based on ... \cite{}.

The analyses select event with an isolated photon candidate passing good quality requirements in the precision regions of the detectors, and a dilepton system with properties compatible with that of the decay of a $Z$ boson. Events are separated according to lepton flavour, the event kinematical properties, and the present of jets compatible with the VBF production of the Higgs boson, in order to maximize the signal sensitivity. The signal is sought for by means of a combined signal-plus-background fit of the photon-dilepton invariant mass spectra in various event categories, where both the continuous background and the signal resonance are parameterized by analytical functions. The Run-2 analyses are strongly driven by statistical uncertainty, and the main systematic uncertainties are from the bias associated to the background modeling.

The extrapolations to HL-LHC are performed with a simple scaling approach, assuming the same signal and background modeling used in the Run-2 analyses. All experimental and systematic uncertainties are considered to remain the same, except the uncertainty associated to the background modeling, which is taken to be negligible.

The ATLAS expected significance to the SM Higgs boson decaying in $Z\gamma$ is 4.9 $\sigma$ with 3000\,$\mathrm{fb}^{-1}$. Assuming the SM Higgs production cross section and decay branching ratios, the signal strength is expected to be measured with a $\pm0.24$ uncertainty. The cross section times branching ratio for the $pp\rightarrow H \rightarrow Z\gamma$ process is projected to be measured as $1.00\pm0.23$ times the SM prediction. Even at the HL-LHC scenario S1, the analysis sensitivity  to \HZy\ will remain driven by the statistical uncertainty. The dominant source of systematic uncertainty in the extrapolation is that associated to the QCD scale variations.

\subsubsection{$H \to ZZ^* \to 4\ell$}
%% {\it To be written by: M. Delmastro}

The measurement of the Higgs boson properties in the \HZZ\ is extrapolated from the most recent measurements by ATLAS with 80\,$\mathrm{fb}^{-1}$ \cite{ATLAS:2018bsg} and CMS with XX\,$\mathrm{fb}^{-1}$ \cite{}.
%%
Events are selected to contain at least two same-flavour opposite-sign dilepton pairs, chosen from isolated electrons and muons candidates passing good quality requirements in the precision regions of the detectors. Additional constraints on the kinematical properties of the pair associated with the decay the on-shell $Z$ boson, and on the global topology of the event, helps to improve the signal to background ratio. The four-lepton invariant mass resolution is improved by correcting for the emission of final-state radiation photons  by the leptons.
%%
The \HZZ\ signal is extracted from the four-lepton invariant mass spectra in the different event categories, after having evaluated the background components using simulation to constrain their shapes, and data control region to extrapolate their normalization in the signal regions.

The dominant systematic uncertainties affecting the extrapolation of the ggH cross section measurement are the lepton reconstruction and identification efficiencies and pile-up modeling uncertainties. The VBF and VH cross-sections are primarily affected by the uncertainty on the jet energy scale and resolution, and by the QCD scale uncertainties. The theory uncertainties related to QCD scale and parton shower modeling primarily affects the extrapolated top cross section.

The VBF, VH and top measurements in the \H2l\ decay channel remain largely dominated by statistical uncertainty when extrapolated to 3000\,$\mathrm{fb}^{-1}$ while the $ggH+b\bar{b}H$ cross section is dominated by systematic uncertainties both in scenario S1 ans S2.
%
Figure~\ref{fig:HZZ_ATLAS_HLLHC_S2} show the ratio of the extrapolated \HZZ\ ATLAS measurements of the main four Higgs production modes to their respective theoretical SM predictions in the scenario S2. The  \HZZ\ HL-LHC measurements are expected to  reach a level of precision comparable to the projected uncertainty on the corresponding theory predictions.

\begin{figure}
  \centering
  \includegraphics[width=0.6\linewidth]{\main/section2/plots/channels/ATLAS_HZZ_compareToSM_prodXS}
  \caption{Ratio of the cross-section times branching fraction measurements of the main four Higgs production modes in the \HZZ\ decay channel to their respective theoretical SM predictions, as extrapolated at the HL-LHC for scenario S2 by ATLAS.}
  \label{fig:HZZ_ATLAS_HLLHC_S2}
\end{figure}

\subsubsection{$H \to WW^* \to \ell\nu\,\ell\nu$}
%% {\it To be written by: M. Delmastro}

The measurement of the Higgs boson properties in the \HZZ\ is extrapolated from the most recent measurements by ATLAS in the \HWW\ channel using 80\,$\mathrm{fb}^{-1}$ \cite{Aaboud:2018jqu}, and by and CMS in the XX channel using XX\,$\mathrm{fb}^{-1}$ \cite{}.
%%
Events are selected to contain two opposite-charged isolated leptons passing good quality requirements in the precision region of the detectors and missing transverse momentum. Additional requirements on the event kinematical properties (e.g. on the dilepton transverse mass and invariant mass, on the azimuthal separation between the leptons) are applied to reduce the various background components. Events are categories as a function of the jet multiplicity, in order to extract the Higgs ggH and VBF production cross sections. The normalization of (non-resonant) $WW$, top ($t\bar{t}$ and $Wt$), and $Z\rightarrow\tau\tau$ backgrounds is set using dedicated control regions of the same jet multiplicity as the signal category to which the normalization is transferred.

The measurements are completely dominated by systematic uncertainties, and their extrapolation to the S2 scenario shows the expected reduction by a factor two. The measurement of the ggH cross section by branching fraction is dominated by theoretical PDF uncertainty, followed by Experimantal uncertainties affecting the signal acceptance, including uncertainties on the jet energy scale and flavour compositing, and lepton misidentification; the VBF result suffers of similar dominant uncertainties.
%%
Figure~\ref{fig:HWW_ATLAS_HLLHC_S2} shows the ratio of the extrapolated \HWW\ ATLAS measurements of the main four Higgs production modes to their respective theoretical SM predictions in scenario S2.

\begin{figure}
  \centering
  \includegraphics[width=0.6\linewidth]{\main/section2/plots/channels/ATLAS_HWW_compareToSM_prodXS}
  \caption{Ratio of the cross-section times branching fraction measurements of the main four Higgs production modes in the \HWW\ decay channel to their respective theoretical SM predictions, as extrapolated at the HL-LHC for scenario S2 by ATLAS.}
  \label{fig:HWW_ATLAS_HLLHC_S2}
\end{figure}
