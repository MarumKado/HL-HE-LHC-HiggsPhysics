\wip{This section will be merged with the next subsection!}\\
The $\kappa$-formalism was introduced in\cite{LHCHiggsCrossSectionWorkingGroup:2012nn,Heinemeyer:2013tqa} as an interim framework to report on the measurements of the Higgs-boson couplings and characterise the nature of the Higgs boson. The $\kappa_{i}$ are defined as ratios of measured cross sections and decay widths with respect to their SM expectation, {\it i.e.}
\begin{equation}
  \label{eq:kappa.EFT.1}
  \kappa^{2}_{X} = \frac{\sigma(X_i\rightarrow h+X_f)}{\sigma(X_i\rightarrow h+X_f)_{\text{SM}}}, \qquad \kappa^{2}_{Y} = \frac{\Gamma(h\rightarrow Y)}{\Gamma(h\rightarrow Y)_{\text{SM}}},
\end{equation}
so that the SM is recovered for $\kappa_i=1$.

The $\kappa$-framework, defined at the level of signal strengths, was appropriate for the observables under study at Run I, which tested deviations in event rates. For Run II and the analyses required at the HL-LHC, differential distributions are needed. In order to study event shapes the formalism, as defined by eq.~\eqref{eq:kappa.EFT.1}, is clearly insufficient and has to be extended.

A closely related issue is how to relate the $\kappa$-framework to a QFT description. A naive interpretation of the $\kappa$ factors as rescalings of SM Higgs couplings has been attempted, but this prescription is not necessarily consistent with QFT principles and has limitations that obstruct a successful implementation. More precisely, the following caveats apply:
\begin{enumerate}
\item In this prescription, only QCD corrections, which are factorisable, can be taken into account. Electroweak corrections cannot be implemented consistently.
\item Gauge invariance and unitarity are generically broken by ad-hoc variations of the SM couplings.
\item  In processes that are loop-induced in the SM, such as $h\to \gamma\gamma$ or $gg\to h$, care has to be taken. A rescaled local coupling, for example for $h\to \gamma\gamma$, does not yield an overall $\kappa_{\gamma}^2$ factor, since the process is not mediated by the local interaction only. In these loop processes the interplay of different couplings, most prominently $\kappa_t$, has to be consistently included.        
\end{enumerate}
The way to circumvent the objections above is to work not at the level of rescaled couplings but at the level of Lagrangians, where locality, unitarity and gauge invariance are automatically implemented. In order to be as general as possible, an upgrade of the $\kappa$-formalism should be embedded in the language of EFTs.

Here we will discuss the interpretation of the $\kappa$ factors within the electroweak chiral Lagrangian (EWChL), also denoted as HEFT in the literature. Within this EFT, and only projecting out the leading contributions to processes with a single Higgs, one finds\cite{Buchalla:2015qju,Buchalla:2015wfa,deBlas:2018tjm}
\begin{align}
  \begin{aligned}
    \label{eq:kappa.EFT.2}
    \mathcal{L}_{\text{fit}} &= 2 c_{V} \left(m_{W}^{2}W_{\mu}^{+}W^{-\mu} +\tfrac{1}{2} m^2_Z Z_{\mu}Z^{\mu}\right) \dfrac{h}{v} - \sum_{\psi}c_{\psi} m_{\psi} \bar{\psi} \psi \dfrac{h}{v} \\
 &+ \dfrac{e^{2}}{16\pi^{2}} c_{\gamma} F_{\mu\nu}F^{\mu\nu} \dfrac{h}{v}+ \dfrac{e^{2}}{16\pi^{2}} c_{Z\gamma} Z_{\mu\nu}F^{\mu\nu} \dfrac{h}{v}+\dfrac{g_{s}^{2}}{16\pi^{2}} c_{g}\langle G_{\mu\nu}G^{\mu\nu}\rangle\dfrac{h}{v},
  \end{aligned}
\end{align}
where $m_{i}$ is the mass of particle $i$, $\psi \in \{t, b, c, \tau, \mu\}$, and the $c_{i}$ describe the modifications of the Higgs couplings.

The previous Lagrangian differs from a naive rescaling of Higgs couplings, even though superficially it might seem to be equivalent. In particular, the Standard Model is consistently recovered in eq.~\eqref{eq:kappa.EFT.2} for
\begin{equation}
  \label{eq:kappa.EFT.3}
    c_{i}^{\text{SM}} = \begin{cases} 1 & \text{for } i = V, t, b, c, \tau, \mu\\ 0 & \text{for } i = g, \gamma, Z\gamma. \end{cases}
\end{equation} 
which is not the case for a naive coupling rescaling. The crucial point is that the coupling modifiers in eq.~\eqref{eq:kappa.EFT.2} are not the full EFT, but just the relevant projection for the processes under study at non-trivial leading order in unitary gauge. Since the couplings differ from the SM ones, the previous Lagrangian, taken in isolation, would be non-renormalisable and break unitarity. These requirements are reinstated once eq.~\eqref{eq:kappa.EFT.2} is understood as part of the EWChL.
 
The EWChL\cite{Dobado:1989ax,Dobado:1989ue,Dobado:1990zh,Dobado:1990jy,Espriu:1991vm,Herrero:1993nc,Herrero:1994iu,Feruglio:1992wf,Bagger:1993zf,Koulovassilopoulos:1993pw,Burgess:1999ha,Wang:2006im,Grinstein:2007iv,Azatov:2012bz,Alonso:2012px,Buchalla:2012qq,Buchalla:2013rka,Buchalla:2013eza} is a bottom-up effective field theory (EFT), constructed with the particle content and symmetries of the SM. These are the same requirements adopted in the construction of the SMEFT. The main difference between both EFTs concerns the Higgs particle. In the EWChL, the Higgs particle, $h$, is included as a scalar singlet, with couplings unrelated to the ones of the Goldstone bosons of EWSB. Therefore, $h$ is not necessarily part of an SU(2) doublet and consequently the leading-order Lagrangian is non-renormalisable, i.e. loop divergences require the addition of new counter-terms. The inclusion of the (finite) number of counter-terms at each loop order makes the theory consistent. The procedure is analogous to the one employed in Chiral Perturbation Theory, whence the name EWChL. Counter-terms needed for the 1-loop renormalisation \cite{Guo:2015isa,Buchalla:2017jlu,Alonso:2017tdy} are included as NLO operators\cite{Buchalla:2013rka} and are therefore suppressed by a loop factor with respect to the leading order. The theory is thus renormalisable order by order in the loop expansion. The embedding of the EFT as a loop expansion can equivalently be expressed as an expansion in chiral dimensions \cite{Buchalla:2013eza}, which allows to identify the counter-terms in a straightforward way. Further details and justifications of the expansion are discussed in \cite{Buchalla:2013rka,Buchalla:2013eza,Buchalla:2015wfa,Buchalla:2016sop}.

Focusing on the leading effects of the measured processes only, the full EWChL reduces to the Lagrangian in eq.~\eqref{eq:kappa.EFT.2}. Note that it includes only single-Higgs processes, as the $\kappa$-formalism also describes only single-Higgs processes. If needed, eq.~\eqref{eq:kappa.EFT.2} can also be extended to describe other processes, simply by projecting the relevant operators already present in the EWChL. For instance, for double-Higgs production from gluon fusion three more operators should be added, corresponding to the interactions $h^{3},\bar{t}th^{2},ggh^{2}$~\cite{Grober:2015cwa,deFlorian:2016spz,Kim:2018uty,Buchalla:2018yce}. Double-Higgs production is discussed in more details in section~\ref{sec:EWChL.double.h}. Since the observed processes are mediated by both tree level and one-loop amplitudes at the first non-vanishing order, operators of leading order in the EFT (first line of eq.~\eqref{eq:kappa.EFT.2}) and next-to-leading order in the EFT (second line of eq.~\eqref{eq:kappa.EFT.2}) have to be included\cite{Buchalla:2015wfa}. Corrections beyond the leading ones, both strong and electroweak, can also be incorporated to arbitrary order in the description of Higgs processes. These corrections involve additional operators, not present in eq.~\eqref{eq:kappa.EFT.2}, but contained in the EWChL.

Understood as corrections to the SM, the $\kappa$ factors can also be generated with the SMEFT (see e.g.\cite{Ghezzi:2015vva} and the discussion in \cite{Brivio:2017vri}). The main differences between both EFT descriptions are the following: (i) in the EWChL, deviations from the SM appear at leading order, and $\mathcal{O}(1)$ corrections to the $\kappa$ factors can be easily accommodated. In the SMEFT, the corrections to the SM appear at NLO, and therefore smaller effects, typically at the percent level, are expected; (ii) In the SMEFT the Higgs is assumed to be a weak doublet. The EWChL instead describes a generic scalar and is therefore closer to the spirit of the $\kappa$ formalism of testing the nature of the Higgs boson.
 
As stated above, the couplings in eq.~\eqref{eq:kappa.EFT.2} can receive a priori large contributions and have to be considered as $\mathcal{O}(1)$ numbers. This is the expectation if new physics contains strongly-coupled new interactions. In some of these scenarios, new-physics interactions can be progressively decoupled from the SM, and it is therefore useful to understand the Wilson coefficients in eq.~\eqref{eq:kappa.EFT.2} as functions of the parameter $\xi = v^{2}/f^{2}$, where $v\approx 246$ \UGeV is the electroweak vacuum expectation value, and $f$ is the scale of new physics. The latter could correspond, for example, to the scale of global symmetry breaking in composite Higgs models. The SM is then recovered for $\xi=0$. For $\xi\ll 1$, one can perform an expansion in $\xi$ on top of the loop expansion in the EWChL. This yields a double expansion in $\xi$ and $1/16\pi^{2}$ \cite{Buchalla:2014eca}, in the spirit of the strongly-interacting light Higgs (SILH) \cite{Giudice:2007fh}. The expected size of the Wilson coefficients is then
\begin{equation}
  \label{eq:kappa.EFT.4}
    c_{i} =  c_{i}^{\text{SM}} + \mathcal{O}(\xi).
\end{equation}
The mapping of the Wilson coefficients $c_{i}$ to the $\kappa_{i}$ parameters is done using the relations of the signal strengths computed from the Lagrangian in eq.~\eqref{eq:kappa.EFT.2}. The necessary formulas can be found in \cite{Buchalla:2015qju,deBlas:2018tjm}. These relations can be written as
%
\begin{equation}
\label{eq:kappa.EFT.5}
  \kappa_{i} =  |f_i(c_{j})| \equiv \frac{|{\cal A}_i(c_{j})|}{|{\cal A}_i(c_{j}^{\text{SM}})|}, 
\end{equation}
%
where ${\cal A}$ is the corresponding transition amplitude of each process. 
The absolute value on the right hand side is necessary, as the loop functions of the light fermions ($b,\tau,\mu,\dots$) for the $\kappa_{\gamma}$ and $\kappa_{g}$ are complex.

The inverse of eq.~\eqref{eq:kappa.EFT.5} is, however, not a well-defined function. We can still obtain an approximate inverse, to connect both formalisms in the opposite direction. This can be easily obtained if we assume that all the imaginary parts are negligible. While this is a good approximation for some of the coefficients in $f_i(c_{j})$, for example for the coefficient of $c_{t}$, it is not the case for the coefficients of the light fermion loops, where real and imaginary parts are of similar size. Nevertheless, as long as the Wilson coefficients stay relatively close to the SM value, neglecting the imaginary parts completely is still a good approximation, because in $\kappa_g$ ($\kappa_\gamma$) the real part of the top loop (top and $W$ loops) contribution dominates over all the other terms.

With the assumption of vanishing imaginary parts, eq.~\eqref{eq:kappa.EFT.5} becomes
\begin{equation}
  \label{eq:kappa.EFT.6}
  \begin{pmatrix}
    \kappa_{V}\\
    \kappa_{t}\\
    \kappa_{b}\\
    \kappa_{\ell}\\
    \kappa_{g}\\
    \kappa_{\gamma}
  \end{pmatrix}
  = 
  \begin{pmatrix}
    1 & 0 & 0 & 0 & 0 & 0 \\
    0 & 1 & 0 & 0 & 0 & 0 \\
    0 & 0 & 1 & 0 & 0 & 0 \\
    0 & 0 & 0 & 1 & 0 & 0 \\
    0 & 1.055 & -0.055 & 0 & 1.3891 & 0 \\
    1.2611 & -0.2683 & 0.0036 & 0.0036 & 0 & -0.3039 \\
  \end{pmatrix}
  \cdot
  \begin{pmatrix}
    c_{V}\\
    c_{t}\\
    c_{b}\\
    c_{\tau}\\
    c_{g}\\
    c_{\gamma}
  \end{pmatrix}.
\end{equation}
%
These numbers also include the leading QCD corrections of the $h\to \gamma\gamma$ and $gg\to h$ amplitude. An explicit comparison of this approximation and the full formulas shows only negligible numerical differences. The inverse of eq.~\eqref{eq:kappa.EFT.6} is
\begin{equation}
  \label{eq:kappa.EFT.7}
  \begin{pmatrix}
    c_{V}\\
    c_{t}\\
    c_{b}\\
    c_{\tau}\\
    c_{g}\\
    c_{\gamma}
  \end{pmatrix}
  = 
  \begin{pmatrix}
    1 & 0 & 0 & 0 & 0 & 0 \\
    0 & 1 & 0 & 0 & 0 & 0 \\
    0 & 0 & 1 & 0 & 0 & 0 \\
    0 & 0 & 0 & 1 & 0 & 0 \\
    0 & -0.76 & 0.04 & 0 & 0.72 & 0 \\
    4.15 & -0.88 & 0.012 & 0.012 & 0 & -3.29 \\
  \end{pmatrix}
  \cdot
  \begin{pmatrix}
    \kappa_{V}\\
    \kappa_{t}\\
    \kappa_{b}\\
    \kappa_{\ell}\\
    \kappa_{g}\\
    \kappa_{\gamma}
  \end{pmatrix}.
\end{equation}
%
With these relations one can translate the results of a $\kappa_i$ fit into the EWChL~formalism and vice-versa. 
In order to do so, however, it is important to have all the relevant information about the fits. In particular,
the median and errors of the parameters are not sufficient, since there may be also significant correlations between them. 