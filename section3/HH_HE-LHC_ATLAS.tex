\begin{center}
\textit{by P. Bokan, E. Petit, M. Wielers}
\end{center}

The results presented in Section~\ref{sec:HH_ATLAS} were extended to provide estimates of the prospects at the HE-LHC, assuming a centre of mass collision energy of 27~$\UTeV$ and 15~$\iab$ of data.

The assumption is made that the detector performance will be the same as of the HL-LHC ATLAS detector. Comparisons between simulation at centre of mass energy of 14 and 27~$\UTeV$ show that the kinematic of the Higgs boson decay particles, as well as the $m_{HH}$ distribution are similar. However the pseudorapidity of the particle tends to point more frequently in the forward region, which would decrease the acceptance by around 10\%. This effect is not taken into account and the impact is expected to be small.

The event yields for the various background processes are scaled by the luminosity increase and the cross-section ratio between the two centre of mass energies. For the signal the cross-section of 139.9 fb is used.

Without including systematic uncertainties a significance of XX and XX standard deviations is expected for the $b\bar{b}\gamma\gamma$ and $b\bar{b}\tau\tau$ channels respectively.
The hypothesis of no Higgs self-coupling can be excluded with a significance of XX and XX standard deviations respectively. Finally the $\kappa_{lambda}$ parameter is expected to be measured with a precision of XX\% and XX\% respectively.