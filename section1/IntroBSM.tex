\subsubsection{Light new physics: rare processes and new degrees of freedom}\label{sec:BSMintro}
A complementary way to unveil BSM physics affecting the Higgs sector of Nature is the search for very rare processes involving the 125 GeV Higgs boson and for extended Higgs sectors. 

The SM predicts several processes involving the Higgs boson to be very rare. Notable examples are the di-Higgs production, as well as the Higgs decays to first and second generation quarks and leptons. The search for these rare processes can unveil the presence of new degrees of freedom. Particularly, measurements of the di-Higgs production cross section (Sec. \ref{sec3}) will give constraints on the Higgs trilinear interaction, therefore providing information on electroweak symmetry breaking and allowing to set constraints on e.g. the nature of the phase transition between the trivial Higgs vacuum and the vacuum we observe at present (Sec. \ref{sec:HH_EWPT}) and on the presence of extended Higgs sectors. The HL and HE stages of the LHC will be crucial to achieve this goal thanks to the relatively sizeable di-Higgs samples that will be produced: $\mathcal O(100~{\rm{K}})$ at HL-LHC and $\mathcal O(2~{\rm{millions}})$ at HE-LHC (compared to the $\mathcal O(6~{\rm{K}})$ di-Higgs produced at Run 1 and 2 LHC). Furthermore, the branching ratios of SM rare Higgs decay modes such as $h\to\mu^+\mu^-$, $h\to Z\gamma$, and $h\to cc$ have been only mildly upper bounded by present LHC searches due in part to the low statistics ($h\to\mu^+\mu^-$, $h\to Z\gamma$) and, in part, to the background limited analyses ($h\to cc$). An important progress on these rare decay modes is expected at the HL and HE-LHC. For example, the HL-LHC will be able to discover and have a $(10-13)\%$ accuracy measurement of the di-muon decay mode (Sec. \ref{Sec:2.3.8}). Knowing the Higgs couplings to light quark and lepton generations will shed light on BSM flavor models and possibly on the SM flavor puzzle (Sec. \ref{sec7}).

Beyond rare SM Higgs processes, BSM models that contain new light degrees of freedom, $X_i$, generically predict rare exotic Higgs, decays $h\to X_i X_j$ or $h\to X_i ~{\rm{SM}}_j$ where ${\rm{SM}}_j$ is a SM particle (Secs. \ref{Sec:6Invisible} and \ref{Sec:9.1Exo}. For a review see e.g. \cite{Curtin:2013fra}). A typical example is the Higgs decaying to light dark matter particles. Thanks to the tiny Higgs width ($\sim 4$ MeV), even very feebly coupled new light particles can lead to relatively sizeable Higgs branching ratios that can be probed by the LHC in the future. On the one hand, the HL and HE-LHC will produce huge samples of Higgs bosons from its main production mode, gluon fusion ($\mathcal O(10^8)$ and $\mathcal O(10^9)$, respectively). This can allow the search for super rare and low background signatures. On the other hand, the sample of Higgs bosons produced from sub-leading  production modes in association with other SM particles (e.g. $tth$) will be sizeable, increasing the discovery prospects for rare and more background limited Higgs decay signatures. 
Therefore, the HL/HE-LHC Higgs exotic decay program can be uniquely sensitive to the existence of a broad range of new light weakly coupled particles (on condition that trigger and analysis thresholds will be kept relatively low, to allow capturing this set of soft signatures).


In many BSM theories, electroweak symmetry is broken not only by one Higgs boson, but by several degrees of freedom. Examples are supersymmetric theories, composite Higgs theories, as well as theories of neutral naturalness. Overall, extended Higgs sectors can lead to new interesting signatures that are not contained in the SM. The search for additional Higgs bosons is a high priority for current and future colliders. The ATLAS and CMS collaborations have performed several searches for heavy neutral and charged Higgs bosons during the first two runs of the LHC. At the same time, the LHCb collaboration (as well as ATLAS and CMS) has pursued several searches for new Higgs bosons with a mass below 125 \UGeV.  The reach of all these searches will expand considerably in the future and, especially, at the HL and HE-LHC. In Secs. \ref{sec:Hff}-\ref{Sec:9.4} and \ref{Sec:9.8} of this report, we study the prospects for testing some of the most promising signatures.
Most of the BSM models that predict the existence of an extended Higgs sector, also predict a 125 \UGeV Higgs with the interactions which are generically different from the SM predictions. As we will show in Secs. \ref{Sec:9.5}-\ref{Sec:9.7}, the study of the interplay between new Higgs searches and Higgs coupling measurements will be a powerful tool to probe vast regions of parameter space of BSM theories with an extended electroweak symmetry breaking sector.