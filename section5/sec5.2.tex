The measurement of the Higgs boson mass by the ATLAS and CMS
experiments at the LHC is~\cite{Tanabashi:2018oca}: $$m_H = 125.18 \pm
0.16 \; \rm GeV$$

This precision is reached with the two high resolution Higgs boson
measurement channels the $H\rightarrow ZZ^* \rightarrow 4\ell$ and
$H\rightarrow \gamma \gamma$. At the LHC Run 2, the precision in the
latter channel is already limited by the systematic uncertainty on the
photon energy scale. The photon energy response is calibrated using
both electrons from Z decays (which requires to be extrapolated from
electrons to photons) and radiative Z events reconstructed with two
charged leptons and a photon, which is limited by statistics in the
transverse momentum range of interest. The most precise measurement is
obtained in the $4\mu$ and the $2e2\mu$ where the on-shell Z mass
constraint can be applied to the $2e$ system. 


Detailed studies of the calibration of the muons, electrons and
photons with the very large HL-LHC sample have not been done yet,
however it is plausible that the mass of the Higgs boson will be
measured with a precision of 10-20 MeV, assuming that with the higher
statistics the analysis will be further optimised to gain in
statistical precision and that systematic uncertainties on the muon
transverse momentum scale will significantly improve with the higher
statistics.
